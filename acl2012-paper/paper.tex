\documentclass[11pt]{article}

\usepackage{acl2012}
\usepackage{times}
\usepackage{latexsym}
\usepackage{amsmath}
\usepackage{multirow}
\usepackage{url}

\usepackage{graphicx}
\usepackage{amsthm}
\usepackage{mathtools}
\usepackage{gnuplottex}

\usepackage{mystyle}

\usepackage[ps2pdf,unicode]{hyperref}   % Musí být za všemi ostatními balíčky

\DeclareMathOperator*{\argmax}{arg\,max}
\setlength\titlebox{6.5cm}    % Expanding the titlebox

\title{TrTok: A Fast and Trainable Tokenizer for Natural Languages}

\author{First Author \\
  Affiliation / Address line 1 \\
  Affiliation / Address line 2 \\
  Affiliation / Address line 3 \\
  {\tt email@domain} \\\And
  Second Author \\
  Affiliation / Address line 1 \\
  Affiliation / Address line 2 \\
  Affiliation / Address line 3 \\
  {\tt email@domain} \\}

\date{}

\begin{document}
\maketitle
\begin{abstract}
We present a data-driven tool for disambiguating token and sentence
boundaries. The implemented system is highly configurable and
versatile to the point its tokenization abilities allow to segment
unbroken Chinese text. The tokenizer relies on maximum entropy
classifiers and requires a sample of tokenized and segmented text as
training data. The system was built with multi-platform libraries only
and with emphasis on speed and correctness. After a necessary survey
of other tools for text tokenization and segmentation and a short
introduction to maximum entropy modelling, a large part of the paper
focuses on the particular implementation we developed and its
evaluation.
\end{abstract}

\section{Introduction}
\label{sec:introduction}

Tokenization and segmentation are parts of almost every natural
language processing system, since in most of the higher-level language
processing applications, words and sentences are the basic processing
units, not streams of bytes encoding characters.

Segmentation (a term we use for what is also referred to as sentence
detection or sentence boundary disambiguation) has been tackled using
a variety of techniques. The most common approaches include writing
heuristics and constructing abbreviation lists (the Stanford
Tokenizer, the RE system) or using machine learning algorithms to
predict the role of a potential sentence terminator (Satz,
MxTerminator, Apache OpenNLP). There have also recently been some very
successful systems using unsupervised methods (Punkt).

Tokenization is a problem which stops being trivial when we start
considering whitespace-free languages such as Chinese or Japanese. In
these languages, tokenization (also referred to as word segmentation)
receives a lot of attention \cite{seg-bakeoff}.

TrTok aims to be a practical tool for tokenizing and segmenting text
written in any language. To achieve such a goal, TrTok relies on the
user determining the specifics of training and tokenization and
providing the necessary training data.

TrTok's novelty comes in the openness and formalization of the
tokenization process and in its resulting general applicability. It is
a continuation of the approach outlined by \citet{sbd-trtok-orig}. The
process is divided into several discrete stages, most of which are
heavily customizable. For example, the user is able to say where in
the text should TrTok consider breaking up or joining tokens or
sentences, how should TrTok represent the context of these decision
points to the underlying classifier, how should the classifier be
trained, how should existing whitespace be treated and more.

TrTok was also built to be a practical tool, which means it can
transparently process text interspersed with XML markup and HTML
entities and was designed to run fast.

The major inconveniences of TrTok are that due to its customizability
it needs to be properly set up and due to its reliance on machine
learning methods, it requires manually tokenized training data.

\section{Previous Work}
\label{sec:previous-work}

Established methods of sentence boundary disambiguation can be
organized into three distinct groups: hand-crafted systems, supervised
learning systems and unsupervised learning systems.

The RE system [] would be an example of a hand-crafted system. The
system scans a document for potential sentence terminators. Once
found, the instance is assumed to be a sentence terminator unless
proven otherwise by a list of regular expressions which examine the
preceding and following tokens for any exceptional situations
(abbreviations, numbers\ldots). This approach drove home the
usefulness of using regular expressions to describe context and the
importance of using abbreviation lists.

MxTerminator [] is a supervised sentence boundary disambiguator using
maximum entropy models to predict whether a potential sentence
terminator does indeed signal the end of a sentence. The features
presented to the classifier are properties of the two tokens
surrounding the potential sentence terminator. The tokens' properties
that are used are the tokens' types, their capitalization and whether
they belong to a list of abbreviations which is either hand-crafted or
induced from test data.

The biggest difference between TrTok and MxTerminator is that TrTok
does not assume any particular selection of features and thus offers
space for richer models (e.g. by extending the width of the context or
providing more complex features like part of speech tags). An example
of a system using more advanced features is the Satz [] system, which
uses possible part of speech tags as features in the machine learning
algorithm.

Unsupervised learning systems are the most distinct from TrTok amongst
all the sentence boundary algorithms as they usually require no manual
configuration nor any training data to function properly. A great
example of an unsupervised sentence boundary disambiguator is the
Punkt system []. It firstly uses collocation detection techniques but
also makes use of an orthographic heuristic to analyze the test data
in several passes and disambiguate abbreviations and sentence
terminators. The system has shown remarkable performance without
needing any manual tuning or training data.

\section{Description of the System}
\label{sec:system}

TrTok is implemented by a parallel execution of several configurable
pipeline steps.

\subsection{TextCleaner}

The first pipeline element, the TextCleaner, reads in and decodes the
input. Additionally, it can strip XML markup from the input and expand
HTML entities to their character counterparts. If the TextCleaner does
so, it can be also configured to relay these changes to the final
pipeline element, the OutputFormatter, which can effectively undo
these changes by reinserting the XML markup and the HTML entities in
the correct position in the output.

\subsection{RoughTokenizer}

The RoughTokenizer stage partitions the stream of characters from the
TextCleaner into small, discrete chunks of non-blank characters,
called rough tokens. The partitioning can be made more granular by
user-defined rules which specify positions at which the desired
tokenization might differ from the whitespace-induced one.

A location in the text may be marked as a \maysplit{} meaning that the
characters in the text preceding and following it may be parts of
different tokens even though they are not separated by whitespace
(e.g. we might wish to put a \maysplit{} between \example{``was''} and
\example{``n't''} in \example{``wasn't''}).

A location within a span of white characters might be labeled as a
\mayjoin{} signalling that the characters preceding and following the
whitespace area might be parts of the same token, as in the case of
spaces entered in long numerals for readability (e.g.
\example{``12~345''}).

Finally, a location in the text may be marked as a \maybreaksentence{}
if the characters preceding it and the characters following it might
belong to different sentences. These locations traditionally occur
after possible sentence terminators, but can include some
domain-specific contexts as well (e.g. line breaks (possibly preceded
by commas) when segmenting poetry).

The placement of the \maysplit{}, \mayjoin{} and \maybreaksentence{}
events further informs the rough tokenization which cuts the rough
tokens so that no rough token contains an event within its boundaries.
Note that the presence of a \may{} event only signifies the
possibility of a tokenization operation (splitting or joining of
tokens or sentences). Whether a token split, token join or sentence
break will occur is up to the Classifier.

As for how the locations of these possible tokenization operations are
determined, each type of event is associated with a set of rules. A
rule consists of a pair of regular expressions, $l$ and $r$. An event
is fired off at a location if a suffix of the text preceding it
matches $l$ and a prefix of the text succeeding matches $r$. All of
the events are handled in this uniform manner with the exception that
the \mayjoin{} patterns also implicitly match the whitespace span
separating the rough tokens.

The most challenging aspect of developing TrTok was to make sure that
the system would be able to correctly identify all of the above events
given any input regular expressions and to do so fast. TrTok solves
this problem by using the lexical analyzer generator Quex (a fast and
more Unicode-friendly variation of the classic tools lex and flex).

On launch, TrTok checks if the files defining the rules for rough
tokenization have changed since last time and if so, constructs a Quex
program by inserting the user-specified regular expressions into a
prepared template. The resulting program is first run through Quex
which turns it into a fast C++ implementation of a FSM which emits
tokens signalling the events in question and transmitting the text in
between them. This code is then compiled, stored for later reuse and
dynamically loaded in. The RoughTokenizer then interfaces with this
code and uses it to process the input it gets from the TextCleaner.

\subsection{FeatureExtractor}

The stream of rough tokens interleaved with potential tokenization
operations output by the RoughTokenizer is processed using the
FeatureExtractor. The FeatureExtractor simply annotates each rough
token with a bit vector signifying which of the user-defined features
hold for the rough token in question.

The features can be defined in two ways: either using a regular
expression or a list of rough tokens. If a feature is defined using a
regular expression, a rough token is said to have that feature iff the
regular expression matches the entire rough token. In the case of a
feature defined using a list of rough tokens, a rough token is said to
have the feature iff it is on the list.

This way it is easy to specify features which try to analyze the shape
of rough tokens using regular expressions or to simply give a list of
all interesting tokens (e.g. words of a certain part of speech or
exceptions such as abbreviations).

\subsection{Classifier}

The Classifier is the other ``hard worker'' of the system (besides the
RoughTokenizer). Its job is to disambiguate the potential tokenization
operations identified by the RoughTokenizer, i.e. it decides whether a
\maysplit{} truly splits a word into two tokens, whether a \mayjoin{}
joins two words into one token and whether a \maybreaksentence{} truly
ends a sentence. It does so by consulting a maximum entropy classifier
for every location containing these potential tokenization operations.

The features passed to the classifier consist of the features of words
in the context surrounding the potential tokenization operation and
the presence of whitespace and potential tokenization operations in
the context area. The user is free to select the size of the context
area and which features from which words in the context area are to be
passed to the classifier.

The maximum entropy module then classifies the instance as either a
SENTENCE\_BOUNDARY, a TOKEN\_BOUNDARY or simply NO\_BOUNDARY. The
Classifier uses this outcome to disambiguate the potential
tokenization operations.

\subsection{OutputFormatter}

The OutputFormatter is the point at which the stream of rough tokens
is turned back into a character stream. This means that all the rough
tokens are concatenated and whitespace is inserted between them
depending on whether there originally was any whitespace between them
and on the tokenization operations which are to be carried out in the
space between them.

Individual tokens end up being separated by a single space character
and sentences are separated by line breaks. Furthermore, instances of
multiple consequent line breaks in the input can be optionally
preserved in the output, which is useful if they were used to separate
the input document into different subdocuments.

On top of that, it can also be the duty of the OutputFormatter to
reinsert any XML markup which was originally present in the text and
to encode characters which originally were HTML entities back into the
same HTML entities. Both of these normalization transformations are
supplied so that idiosyncrasies of the coding and any XML metadata do
not interfere with the feature extraction or token boundary
classification.

\subsection{Encoder}

The final stage of the process is the Encoder whose only
responsibility is to convert the internal UTF\-8-encoded character
stream into the target encoding and to send it to the program's output.

\subsection{Modes of Operation}

The pipeline described above applies when we are using TrTok to
tokenize data. However, before we can do that, we need to train a
classifier for the Classifier stage. For this purpose, TrTok offers
another mode of operation for training which uses a collection of raw
input data paired with the manually tokenized and segmented data to
train the model.

Besides these two core modes, TrTok also includes a mode for
tokenizing data and then comparing the results with manually annotated
data and outputting the difference so that its performance can be
evaluated. Finally, TrTok can also ``tokenize'' a file without a
classifier, choosing a sentence or token boundary wherever possible to
prepare data more suitable for manual annotation.

\subsection{Tokenization Schemes}

The configuration of the system (the behavior of the pipeline stages
all the way from the TextCleaner to the OutputFormatter) is called a
\newterm{tokenization scheme}. TrTok is built with support for a
hierarchical organization of tokenization schemes where tokenization
schemes lower in the hierarchy share the configuration files of those
higher in the hierarchy unless they provide their own version. This
cuts down duplication in the configuration of the tool for different
languages or contexts and makes it possible to tweak an existing
scheme without modifying it by creating a subscheme.

\section{Evaluation}
\label{sec:eval}

We evaluated our implementation of TrTok compared to other prominent
implementations and approaches to sentence detection. The included
competitors range from the simple rule-based RE system through the
maximum entropy powered approaches of the MxTerminator and of Apache
OpenNLP's SentenceDetector all the way to the unsupervised Punkt
system and the hand-crafted PTBTokenizer and sentence splitter used in
Stanford CoreNLP. The results are presented in
Table~\ref{tbl:grand-melee}.

\subsection{Dataset}

The experiments were conducted on the Brown corpus as is supplied
through NLTK. A representative (covering each category of text
proportionately) 20\% of the corpus was used as the testing data. This
number was chosen so that the testing data would be sure to contain at
least 1000 instances of a non-sentence-terminating full stop, the
resulting test set contains 1481 such full stops. The rest of the data
was made available for training to the supervised learning methods.

\subsection{Sentence Detection Methods}

Absolute Baseline is simply the approach which marks every full stop as
a sentence terminator.

Trtok::Baseline is the simplest tokenizer which can be written in
TrTok. The fact that this performs noticeably better than the Absolute
Baseline is due to the fact that even if we do not specify any
contextual features explicitly, TrTok still passes the built-in
feature 0:\%WHITESPACE to the classifier, which tells it whether the
full stop in question is followed by any whitespace or not. The
classifier has then learned that periods not followed by whitespace
usually do not mean the end of a sentence and uses this to perform
better than the Absolute Baseline.

TrTok::Satz-like is a straightforward attempt at translating the Satz
system to TrTok. The part-of-speech-tagged training data was used to
construct lexicons for each different part of speech tag (NLTK's
method of simplifying tags was used to reduce the number of different
tags to help fight data sparsity). The classifier was then given all
the retrieved part of speech tags for the 3 rough tokens preceding and
succeeding a full stop. In contrast to Satz, our version uses a
different system of tags, a different machine learning algorithm and
most importantly our version does not try to guess the part of speech
tags for words which are not found in the lexicon.

The RE system, MxTerminator and Punkt were described in
Section~\ref{sec:previous-work}. The implementation of RE used for our
experiments was the one provided at
\texttt{http://www.ppgia.pucpr.br/∼silla/softwares/yasd.zip}, the
implementation of MxTerminator was obtained from
\texttt{ftp://ftp.cis.upenn.edu/pub/adwait/jmx/jmx.tar.gz} and the
implementation of Punkt is the one from NLTK. As for training, Punkt
received the entire Brown corpus (training data and testing data)
without any annotations while MxTerm was trained using the training
data.

Apache OpenNLP contains a sentence detector based around a maximum
entropy classifier. The implementation is nearly identical to the
specification of MxTerminator with only minor deviations (such as
signalling whitespace around the full stop as features).

We performed experiments both with the ready-to-use sentence detection
model for English distributed via OpenNLP's website and with a model
which was trained on our training data. The parameters for optimally
training the model were estimated using 10-fold cross-validation on
the training data.

The Stanford CoreNLP sentence splitter works by applying its tokenizer
to the input text which will make the distinction between a full stop
as part of an abbreviation or an ordinal number as opposed to being a
sentence terminator. Thus the task of sentence splitting is trivial
after the tokenization has been correctly performed.

The tokenization itself is implemented using a lexical analyzer
generator, JFlex (not unlike how TrTok uses Quex to implement the
RoughTokenizer). The tokenizer itself is a deterministic program
guided by a collection of heuristics.

TrTok::MxTerm-like is a translation of the MxTerminator method to
TrTok. Whereas MxTerm inspects the prefix and suffix of the
full-stop-containing word and the words preceding and succeeding it,
TrTok::MxTerm-like splits the period into its own rough token and
examines the two rough tokens on either side of it. Also,
TrTok::MxTerm-like does

The reason why MxTerminator does not achieve the same performance
could be that the maximum entropy trainer used in MxTerminator limits
itself to 100 iterations of Generalized Iterative Scaling, which
converges very slowly compared to L-BFGS. Another reason might be the
fact that both MxTerminator and OpenNLP use cut off infrequent
features. Also, both OpenNLP and MxTerminator try to induce
abbreviations from training data whereas TrTok::MxTerm-like has a
collection of abbreviations which was obtained by taking the union of
all the different abbreviation lists used in TrTok::Groomed.

Finally, TrTok::Groomed is a relatively large hand-made tokenization
scheme. It includes 7 distinct lists of abbreviations (prefix and
suffix titles, abbreviated names of months\ldots) totalling 303 types,
properties for detecting the case of tokens, for noticing numbers
which happen to be in the range of the days of the month or years,
etc\ldots. These features are extracted from all rough tokens within
the 8 tokens from the full stop and for the 2 closest tokens on either
side of the full stop, the token's type itself is passed as a feature.

Don't forget to mention how the performance measurement were taken!

Placeholder!\cite{sbd-punkt}

\begin{table*}
  \begin{center}
    \begin{tabular}{ | l | c | c | c | c | c | }
      \hline
      & Accuracy $\downarrow$ & Error Rate & Precision
      & Recall & F_1 Measure \\ \hline
      TrTok::Groomed & \textbf{98.87\%} & \textbf{1.12\%} & \textbf{99.13\%}
                     & 99.57\% & \textbf{99.35\%} \\ \hline
      TrTok::MxTerm-like & 98.76\% & 1.24\% & 98.70\%
                         & 99.89\% & 99.29\% \\ \hline
      Punkt & 98.63\% & 1.36\% & 98.81\%
            & 97.35\% & 98.08\% \\ \hline
      Stanford CoreNLP & 98.46\% & 1.53\% & 98.76\%
                       & 99.48\% & 99.12\% \\ \hline
      MxTerminator & 98.10\% & 1.89\% & 98.30\%
                   & 96.68\% & 97.48\% \\ \hline
      Apache OpenNLP & 97.97\% & 2.02\% & 98.20\%
                     & 96.30\% & 97.24\% \\ \hline
      Apache OpenNLP (pre-trained) & 97.47\% & 2.52\% & 98.62\%
                                   & 89.97\% & 94.09\% \\ \hline
      RE & 97.11\% & 2.88\% & 98.66\%
         & 87.33\% & 92.65\% \\ \hline
      TrTok::Satz-like & 96.50\% & 3.49\% & 97.90\%
                      & 98.07\% & 97.99\% \\ \hline
      TrTok::Baseline & 91.84\% & 8.15\% & 91.67\%
                      & 99.66\% & 95.50\% \\ \hline
      Absolute Baseline & 86.89\% & 13.11\% & 86.89\%
                        & \textbf{100.00\%} & 92.99\% \\ \hline
    \end{tabular}
  \end{center}
  \caption[Performance of sentence detectors on the Brown corpus]
    {The performance of the various sentence detectors on the Brown corpus.}
  \label{tbl:grand-melee}
\end{table*}

\section{Conclusion}
\label{sec:outro}

We have presented and described a universal tool for segmenting and
tokenizing textual data. We have applied the tool to detecting
sentences in English text and identifying words in Chinese text. We
have shown that in both cases, TrTok can offer performance which is
competitive with previous approaches, more so in the case of English
sentence detection, where TrTok outperformed existing implementations.

Since TrTok lets us define a lot of its behavior using declarative
rules and feature descriptions, it might be interesting to harness
this ability to find out the effect of various contextual cues on the
performance of a sentence detector.

On the software side of things, TrTok would also benefit from getting
more user-friendly, which would entail providing a walkthrough of the
setup process, distributing example setups and trained models and
offering an all-dependencies-included compiled package for easier
deployment.


\bibliographystyle{acl2012}
\bibliography{sbd,maxent,seg,web,data}

\end{document}
