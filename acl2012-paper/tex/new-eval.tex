\section{Evaluation}
\label{sec:eval}

We evaluated our implementation of TrTok compared to other prominent
implementations and approaches to sentence detection. The included
competitors range from the simple rule-based RE system through the
maximum entropy powered approaches of the MxTerminator and of Apache
OpenNLP's SentenceDetector all the way to the unsupervised Punkt
system and the hand-crafted PTBTokenizer and sentence splitter used in
Stanford CoreNLP. The results are presented in
Table~\ref{tbl:grand-melee}.

\begin{table*}
  \begin{center}
    \begin{tabular}{ | l | c | c | c | c | c | }
      \hline
      & Accuracy $\downarrow$ & Error Rate & Precision
      & Recall & F_1 Measure \\ \hline
      TrTok::Groomed & \textbf{98.85\%} & \textbf{1.14\%} & \textbf{99.11\%}
                     & 99.57\% & \textbf{99.34\%} \\ \hline
      TrTok::MxTerm-like & 98.76\% & 1.23\% & 98.70\%
                         & 99.88\% & 99.29\% \\ \hline
      TrTok::Easy & 98.69\% & 1.30\% & 98.61\%
                  & 99.90\% & 99.25\% \\ \hline
      Punkt & 98.64\% & 1.35\% & 98.81\%
            & 99.63\% & 99.22\% \\ \hline
      Stanford CoreNLP & 98.53\% & 1.46\% & 98.75\%
                       & 99.57\% & 99.16\% \\ \hline
      MxTerminator & 98.27\% & 1.72\% & 98.30\%
                   & 99.73\% & 99.01\% \\ \hline
      Apache OpenNLP & 98.20\% & 1.79\% & 98.19\%
                     & 99.76\% & 98.97\% \\ \hline
      Apache OpenNLP (pre-trained) & 97.70\% & 2.29\% & 98.61\%
                                   & 98.74\% & 98.68\% \\ \hline
      RE & 97.25\% & 2.74\% & 98.52\%
         & 98.31\% & 98.41\% \\ \hline
      TrTox::Satz-like & 96.50\% & 3.49\% & 97.90\%
                       & 98.07\% & 97.99\% \\ \hline
      TrTok::Baseline & 91.84\% & 8.15\% & 91.67\%
                      & 99.66\% & 95.50\% \\ \hline
      Absolute Baseline & 86.89\% & 13.10\% & 86.89\%
                        & \textbf{100.00\%} & 92.98\% \\ \hline
    \end{tabular}
  \end{center}
  \caption[Performance of sentence detectors on the Brown corpus]
    {The performance of the various sentence detectors on the Brown corpus.}
  \label{tbl:grand-melee}
\end{table*}

\subsection{Dataset}

The experiments were conducted on the Brown corpus as is supplied
through NLTK. A representative (covering each category of text
proportionately) 20\% of the corpus was used as the testing data. This
number was chosen so that the testing data would be sure to contain at
least 1000 instances of a non-sentence-terminating full stop; the
resulting test set ended up containing 1481 such full stops. The rest
of the data was made available for training to the supervised learning
methods.

\subsection{Performance Measurement}

The performance of the evaluated systems was measured by their success
in classifying instances of the full stop character. The text contains
other sentence terminators such as the question mark and the
exclamation mark, but they almost never figure as anything else but
sentence terminators in the text. Other occasional sentence delimiters
such as dashes, semicolons, colons and line breaks (e.g. in headlines)
were ignored as well, since the other systems usually do not have a
solution for them. This way the comparison is fair. Furthermore, the
full stop is the most common and ambiguous of the sentence delimiters.

\subsection{Sentence Detection Methods}

Absolute Baseline is simply the approach which marks every full stop as
a sentence terminator.

Trtok::Baseline is the simplest tokenizer which can be written in
TrTok. The reason that this performs noticeably better than the
Absolute Baseline is that even if we do not specify any contextual
features explicitly, TrTok still passes the built-in feature
\texttt{0:\%WHITESPACE} to the classifier, which tells it whether the
full stop in question is followed by any whitespace or not. The
classifier has then learned that periods not followed by whitespace
usually do not mean the end of a sentence and uses that to perform
better than the Absolute Baseline.

TrTok::Satz-like is a straightforward attempt at translating the Satz
system to TrTok. The POS-tagged training data was used to construct
lexicons for each different part of speech tag (NLTK's method of
simplifying tags was used to reduce the number of different tags to
help fight data sparsity). The classifier was then given all the
retrieved part of speech tags for the 3 rough tokens preceding and
succeeding a full stop.

In contrast to Satz, our version uses a different system of tags, a
different machine learning algorithm and most importantly our version
does not try to guess the part of speech tags for words which are not
found in the lexicon. All of these decisions are for the worse (except
maybe for the choice of machine learning algorithm and hypothesis
representation) and so TrTok::Satz-like does not perform as well as
the original Satz system would. Sadly, the implementation of Satz is
no longer available.

The RE system, MxTerminator and Punkt were described in
Section~\ref{sec:previous-work}. The implementation of RE used for our
experiments was the one provided
here\footnote{\texttt{http://www.ppgia.pucpr.br/∼silla/softwares/yasd.zip}},
the implementation of MxTerminator was obtained from
here\footnote{\texttt{ftp://ftp.cis.upenn.edu/pub/adwait/jmx/jmx.tar.gz}}
and the implementation of Punkt is the one from NLTK. As for training,
Punkt received the entire Brown corpus (training data and testing
data) without any annotations while MxTerm was trained using the
training data.

Apache OpenNLP contains a sentence detector based around a maximum
entropy classifier. The implementation is nearly identical to the
specification of MxTerminator with only minor deviations (such as
signalling whitespace around the full stop as features).

We performed experiments both with the ready-to-use sentence detection
model for English distributed via OpenNLP's
website\footnote{http://opennlp.sourceforge.net/models-1.5/en-sent.bin}
and with a model which was trained on our training data. The
parameters for optimally training the model were estimated using
10-fold cross-validation on the training data.

The Stanford CoreNLP sentence splitter works by applying its tokenizer
to the input text which will make the distinction between a full stop
as part of an abbreviation or an ordinal number as opposed to being a
sentence terminator. Thus the task of sentence splitting is trivial
after the tokenization has been correctly performed.

The tokenization is implemented using a lexical analyzer generator,
JFlex (not unlike how TrTok uses Quex to implement the
RoughTokenizer), it is a deterministic program guided by a collection
of heuristics.

TrTok::MxTerm-like is a translation of the MxTerminator method to
TrTok. Whereas MxTerm inspects the prefix and suffix of the
full-stop-containing word and the words preceding and succeeding it,
TrTok::MxTerm-like splits the period into its own rough token and
examines the two rough tokens on either side of it.

TrTox::MxTerm-like also nicely demonstrates the ease with which new
tokenization schemes can be defined. The entire setup consisted of
creating a directory, collecting the abbreviations in a single file
and writing 5 lines of configuration, 2 or 3 of which could be easily
obsoleted by adopting saner defaults in TrTok and 1 of which is purely
for convenience.

The reason why MxTerminator does not achieve the same performance
could be that the maximum entropy trainer used in MxTerminator limits
itself to 100 iterations of Generalized Iterative Scaling, which
converges very slowly compared to L-BFGS. Another reason might be the
fact that both MxTerminator and OpenNLP use cut off infrequent
features. Also, both OpenNLP and MxTerminator try to induce
abbreviations from training data whereas TrTok::MxTerm-like has a
prepared collection of abbreviations which was obtained by taking the
union of all the different abbreviation lists used in TrTok::Groomed.

The high performance of TrTok::MxTerm-like led us to try and see what
happens when we simplify the tokenization scheme further, which led to
TrTok::Easy which works the same way as TrTok::MxTerm-like, but it
does not use any abbreviation lists, merely the token types
surrounding the full stop. Therefore, TrTok::Easy does not rely on any
external linguistic knowledge and should therefore be fairly language
universal, given that we have enough training data. The performance of
TrTok::Easy also points out that the difference in performance between
TrTok and MxTerminator/OpenNLP could not be explained by the different
abbreviation lists.

Finally, TrTok::Groomed is a relatively large hand-made tokenization
scheme. It includes 7 distinct lists of abbreviations (prefix and
suffix titles, abbreviated names of months\ldots) totalling 303 types,
properties for detecting the case of tokens, for noticing numbers
which happen to be in the range of the days of the month or years,
etc\ldots. These features are extracted from all rough tokens within
the 8 tokens from the full stop and for the 2 closest tokens on either
side of the full stop, the token's type itself is passed as a feature.

Interestingly, there is not much difference in the performance of
TrTok::Groomed, TrTok::MxTerm-like and TrTok::Easy. This tells us that
besides the token types in the close vicinity of the full stop, other
features are not that important. This highlights another useful use
for TrTok as a tool for the fast analysis of the importance of
different contextual features for performing the task of sentence
detection.

\section{Chinese Word Segmentation}

Since TrTok is basically a general program for splitting text into
sequences (sentences) which are in turn composed of other sequences
(tokens) based on user-defined contextual features, TrTok can be used
more than just for sentence detection. The task of splitting text into
tokens is a fairly trivial task in many Western languages, but it
poses a real challenge in languages which do not separate their words
explicitly by spaces, such as Chinese.

We ported the key features of one of the top contestants (which also
happened to employ a maximum entropy classifier) in the 2005 Second
International Chinese Word Segmentation Bakeoff happened to use
maximum entropy models, into TrTok and evaluated its performance using
the bakeoff evaluation scripts. The resulting segmenter achieved a
median performance with respect to the other contestants of the
Bakeoff.

Placeholder!\cite{sbd-punkt}
