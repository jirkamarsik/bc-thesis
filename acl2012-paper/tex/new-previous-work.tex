\section{Previous Work}
\label{sec:previous-work}

Established methods of sentence boundary disambiguation can be
organized into three distinct groups: hand-crafted systems, supervised
learning systems and unsupervised learning systems.

The RE system [] would be an example of a hand-crafted system. The
system scans a document for potential sentence terminators. Once
found, the instance is assumed to be a sentence terminator unless
proven otherwise by a list of regular expressions which examine the
preceding and following tokens for any exceptional situations
(abbreviations, numbers\ldots). This approach drove home the
usefulness of using regular expressions to describe context and the
importance of using abbreviation lists.

MxTerminator [] is a supervised sentence boundary disambiguator using
maximum entropy models to predict whether a potential sentence
terminator does indeed signal the end of a sentence. The features
presented to the classifier are properties of the two tokens
surrounding the potential sentence terminator. The tokens' properties
that are used are the tokens' types, their capitalization and whether
they belong to a list of abbreviations which is either hand-crafted or
induced from test data.

The biggest difference between TrTok and MxTerminator is that TrTok
does not assume any particular selection of features and thus offers
space for richer models (e.g. by extending the width of the context or
providing more complex features like part of speech tags). An example
of a system using more advanced features is the Satz [] system, which
uses possible part of speech tags as features in the machine learning
algorithm.

Unsupervised learning systems are the most distinct from TrTok amongst
all the sentence boundary algorithms as they usually require no manual
configuration nor any training data to function properly. A great
example of an unsupervised sentence boundary disambiguator is the
Punkt system []. It firstly uses collocation detection techniques but
also makes use of an orthographic heuristic to analyze the test data
in several passes and disambiguate abbreviations and sentence
terminators. The system has shown remarkable performance without
needing any manual tuning or training data.
