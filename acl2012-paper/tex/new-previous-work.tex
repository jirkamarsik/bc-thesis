\section{Previous Work}
\label{sec:previous-work}

Established methods of sentence boundary disambiguation can be
organized into three distinct groups: hand-crafted systems, supervised
learning systems and unsupervised learning systems.

The \textbf{RE} system \cite{sbd-re} is an example of a hand-crafted
system. The program scans a document, looking for full stops. When one
is found, the word preceding it is compared to a list of regular
expression exceptions (mostly abbreviations) and unless the word is
found to match one of them, it is assumed to end the sentence. Besides
this core, the system also implements a small heuristic which checks
for numbers preceding the full stop and the word following it.

\textbf{MxTerminator} \cite{sbd-mxterm} is a supervised sentence
boundary disambiguator using maximum entropy models to predict whether
a potential sentence terminator does indeed signal the end of a
sentence. The prefix and suffix of the word containing the potential
sentence terminator and the words preceding and following it are
analyzed and their features are passed to the classifier. The features
consist of the tokens' type, their capitalization and their membership
status on a list of abbreviations which are either hand-prepared or
induced from data.

The biggest difference between TrTok and MxTerminator is that TrTok
does not assume any particular selection of features and thus offers
space for richer models (e.g. by extending the width of the context or
providing more complex features like part of speech tags).

An example of a system using more advanced features is the
\textbf{Satz} system \cite{sbd-satz}, which uses possible part of
speech tags as features in the machine learning algorithm.

Unsupervised learning systems are the most distinct from TrTok amongst
all the sentence boundary detection algorithms as they usually require
no manual configuration nor any training data to function properly. A
great example of an unsupervised sentence boundary disambiguator is
the Punkt system \cite{sbd-punkt}.

\textbf{Punkt} relies mostly on collocation detection techniques but
also makes use of an orthographic heuristic to analyze the test data
in several passes and disambiguate abbreviations and sentence
terminators. The system has shown remarkable performance without
needing any manual tuning or training data.
