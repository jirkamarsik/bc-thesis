\chapter*{Introduction}
\addcontentsline{toc}{chapter}{Introduction}

The goal of this thesis was to provide a fast implementation of a system for
disambiguating token and sentence boundaries and to evaluate the
implementation both in terms of its accuracy and its speed.

Token and sentence boundary disambiguation may seem trivial at first, and it
usually is, but in some occasions it might turn out to be quite complex.
Consider the following cases:

\begin{exe}
  \ex{\label{ex:long-context-a}
      On Friday, the 22\textsuperscript{nd}, at around 2 a.m.\ Dr.~T.~Adams
      finished the preliminary examination.}
  \ex{\label{ex:long-context-b}
      The field tests were to begin on Friday, the 22\textsuperscript{nd}, at
      around 2 a.m. Dr.~T.~Adams finished the preliminary examination the night
      before.}
  \ex{"314 159.26\$, about half of the yearly budget, was spent on office
      redecoration!", protested the disgruntled employee of Vanity, S.p.A\@.}
\end{exe}

Even as I was typesetting these examples in \LaTeX{}, I had to explicitly mark
some of the periods in the above examples as not being sentence boundaries, as
\LaTeX{} likes to insert slightly larger spaces after sentence terminators (so
called French spacing). The heuristic used by \LaTeX{} is very simple: if a
word-final potential sentence terminator (a period, a question mark or an
exclamation mark) follows a capital letter, then it is most likely a part of
an abbreviation (or an initial) and so it does not mark the end of a
sentence\footnote{A more intuitive approach might be to check the case of the
following, not the preceding, word.} \cite{web-latex}.

Such a simple system runs into problems in the examples given above, as we
can see that abbreviations do not necessarily end with capital letters and on
top of that, a period may serve both as part of an abbreviation and as a
sentence terminator. Examples \ref{ex:long-context-a} and
\ref{ex:long-context-b} also show us that the context needed to disambiguate
the sentence boundary may be quite far from the boundary in question.

While getting the size of a space correctly down to the last millimeter is
certainly a noble goal, there are also some important uses for a more reliable
segmenter and tokenizer. When text is being processed and parsed by automatic
tools, a common first step is to divide the text into tokens and sentences. A
lot of the tools that then work with these tokens assume they are correct and
try to analyze them further. As a lot of these tools are getting more and more
accurate, it is important we step up the quality of the tokenization process,
so that the system's quality is not determined by something as basic as
tokenization and segmentation of input. 

In the last 20 years, the problem started getting some recognition and several
systems were demonstrated. This thesis does not aim to create a new system for
tokenization. This work is based on an already existing tokenizer implemented
by Ondřej Bojar during the construction of the UMC 0.1 Czech-Russian-English
Multilingual Corpus \cite{maxent-original-paper,maxent-original}.

A key feature of the original tokenizer is its strict segregation of
language-dependent knowledge into configurable files. The new implementation
expands on this idea and assumes next to nothing about the language being
processed except that the sentence and token boundaries are disambiguated by a
limited context window described by binary predicates expressed as regular
expressions. The tokenizer thus offers a great deal of customizability and a
lot of effort has been put into ensuring that the tokenizer will behave as
expected and that the behaviour is easy to understand without diverging too
much from the original.

Performance, being the motivation behind the current implementation, was also
important. Both the original and the new tokenizer rely on a C++ toolkit which
handles the mechanics of machine learning \cite{maxent-toolkit}. However, the
original implementation, being written in Perl, had to access the functionality
through a command-line interface passing data through files. The new
implementation will have the benefits of using the C++ API directly. Where the
old implementation used regular expressions to partition the input and detect
potential token and sentence boundaries, the new implementation uses a lexical
analyzer generator \cite{web-quex} to generate fast C++ code, compile it and
load it at runtime. The new implementation also benefits from the multiple CPUs
found on modern computers and uses a high-level parallelism library
\cite{web-tbb} to perform the various time-consuming tasks of tokenization in
parallel.

In Chapter~\ref{chap:survey}, we will look at other systems which tried to
tackle the problem and compare them to our tokenizer. In
Chapter~\ref{chap:maxent}, a brief overview of the maximum entropy method of
machine learning will be given. Chapter~\ref{chap:impl} will familiarize us
with the implementation of the tokenizer. Finally, in Chapter~\ref{chap:eval},
we evaluate the speed and accuracy of the tokenizer on several datasets.
