\chapter{Evaluation}
\label{chap:eval}



\section{The Accuracy of the System}
\label{sec:eval-acc}

\subsection{Chinese Word Segmentation}
\label{ssec:eval-acc-chinese}

Tokenizing latin-script languages is not very hard. We can usually get by well
enough by splitting the text at whitespaces and at boundaries between different
classes of symbols. Sometimes, we might want to be more specific and try to
tokenize English contractions as separate words. However, these problems are
quite easy to solve when compared to the task of tokenizing Chinese text. The
absence of any spaces between words forbids the use of any simple heuristic and
linguistically empowered methods must be used.

We took inspiration from the system for Chinese word segmentation presented in
Section~\ref{survey-chinese} which is also based on maximum entropy models. The
basic features used in that system were ported to our formalism. The biggest
difference between the systems was the fact that the original Chinese tokenizer
classified individual characters as being single words or the beginnings,
middles or ends of a multi-character word. Our tokenizer is constrained by our
fomalism and so we must immediately classify each character boundary whether it
forms an token boundary.

We were able to obtain the same data on which the original tokenizer was
developed, which happen to be the training data for the Second International
Chinese Word Segmentation Bakeoff. The bakeoff was a competition challenging
computational linguists to develop word segmentation systems for Chinese using
the supplied data for training. The provided data consists of 4 datasets
provided by Academia Sinica, City University of Hong Kong, Peking University
and Microsoft Research. Each of these datasets adopts slightly different
tokenization standards and so we train and test our tokenizer on the datasets
individually. Each dataset comes with a training part and a testing. We
strictly used only the training part when developing our tokenizer and used the
testing part only at the end to evaluate our results. The only thing we knew
about the testing data in advance was its size which helped us choose a
reasonable size for our heldout data.

First off, we split our training data into a development part and a heldout
part. We chose the size of the heldout data to be roughly as big as the testing
data so we could trust our system's performance on it to be representative of
our system's true accuracy. The sizes of the partitioned datasets can be seen
in Table~\ref{tbl:bakeoff-sizes}.

\begin{table}
  \begin{center}
    \begin{tabular}{ | l | r | r | r | }
      \hline
      & \multicolumn{2}{ | c | }{Training data} & Testing data \\ \hline
      & Development data & Heldout data & Testing data \\ \hline
      Academia Sinica & 39686533 & 1057344 & 942571 \\ \hline
      City University & 8283540 & 266129 & 240767 \\ \hline
      Peking University & 7008808 & 719430 & 718331 \\ \hline
      Microsoft Research & 16100177 & 791333 & 766786 \\
      \hline
    \end{tabular}
  \end{center}
  \caption[Bakeoff dataset sizes]{The sizes of the individual parts of the bakeoff datasets in bytes.}
  \label{tbl:bakeoff-sizes}
\end{table}

We set the event cutoff to 2 as in [], so we retain a lot of the encountered
bigrams but still keep the number of parameters manageable. We then
experimented with training the tokenizer and testing it on the heldout data.
Depending on how much we constrained training time, the tokenizer could either
be undertrained or overfitted. The heldout data served as an independent
indicator telling us how close we are to the ideal balance between a detailed
and a general model. Experimentation led us to restrain the number of training
iterations to the values seen in Table~\ref{tbl:bakeoff-iters}. We can see that
the suitable number of iterations spent training is nearly a linear function of
the dataset size. This stems from the fact that a larger dataset means more
bigrams and unigrams and thus more parameters to estimate.

\begin{table}
  \begin{center}
    \begin{tabular}{ | l | r | }
      \hline
      & Number of iterations \\ \hline
      Academia Sinica & 1000 \\ \hline
      City University & 175 \\ \hline
      Peking University & 150 \\ \hline
      Microsoft Research & 400 \\
      \hline
    \end{tabular}
  \end{center}
  \caption[Recommended iterations for Chinese segmentation]{A suitable number of iterations for training on a given dataset in
           the bakeoff data.}
  \label{tbl:bakeoff-iters}
\end{table}

After we established the training parameters, we trained the system on the
entier training data and checked its performance on the gold testing data. The
performance of the development system on the heldout data and of the final
system on the testing data can be seen in Tables \ref{tbl:bakeoff-devel} and
\ref{tbl:bakeoff-final}.

\begin{table}
  \begin{center}
    \begin{tabular}{ | l | r | r | r | r | }
      \hline
      & Accuracy & Precision & Recall & F-measure \\ \hline
      Academia Sinica & 95.55\% & 96.38\% & 95.78\% & 96.08\% \\ \hline
      City University & 91.58\% & 94.28\% & 91.67\% & 92.95\% \\ \hline
      Peking University & 92.70\% & 94.20\% & 93.70\% & 93.95\% \\ \hline
      Microsoft Research & 93.19\% & 95.71\% & 92.79\% & 94.22\% \\
      \hline
    \end{tabular}
  \end{center}
  \caption[Development performance of Chinese segmenter]{The performance of the system trained on the development data when
           tokenizing the heldout data.}
  \label{tbl:bakeoff-devel}
\end{table}

\begin{table}
  \begin{center}
    \begin{tabular}{ | l | r | r | r | r | }
      \hline
      & Accuracy & Precision & Recall & F-measure \\ \hline
      Academia Sinica & 93.82\% & 94.63\% & 94.92\% & 94.77\% \\ \hline
      City University & 91.75\% & 94.38\% & 91.62\% & 92.98\% \\ \hline
      Peking University & 91.94\% & 94.94\% & 91.43\% & 93.15\% \\ \hline
      Microsoft Research & 94.34\% & 96.19\% & 93.79\% & 94.98\% \\
      \hline
    \end{tabular}
  \end{center}
  \caption[Final performance of Chinese segmenter]{The performance of the system trained on the entire training data
           when tokenizing the gold testing data.}
  \label{tbl:bakeoff-final}
\end{table}

While the resulting adapted system does not perform as well as the original
word segmenter by Low, Ng and Guo, it achieves a median performance compared to
the performance of the other bakeoff submissions. The result is quite pleasing,
given that the all we needed was to write the feature definitions into a few
files and toy with some training parameters.

\subsection{Tokenization of Czech and English}
\label{ssec:eval-acc-eng}

For evaluating the accuracy of tokenizing Czech and English text, four
different methods were implemented. The Absolute Baseline relies on no other
piece of information than the current decision point and the whitespace
following it to classifiy boundaries. It is there to show the minimum possible
line every tokenizer should pass. The Simple Tokenizer checks for capital
letters before the period (initials) and after the period (start of a new
sentence). It represents the often too simple approach to tokenization. The
English-only Satz-like system uses only part of speech data about the
surrounding tokens to predict a boundary. Finally, the Groomed Tokenizer is the
tokenization scheme used in the original reference implementation, which has
been supplied with lists of abbreviations and lots of useful regular
expressions.

All systems were tested both on a sample of data from CzEng and, in the case of
the English tests, also on the Brown corpus. All datasets were divided into
equally large development, heldout and testing sets to be used as in
Section~\label{ssec:eval-acc-chinese}. As for the part of speech data of the
Satz-like system, lexicons for each part of speech were extracted from the
training section of the Brown corpus for the Brown corpus exercise and from the
entire Brown corpus for the CzEng exercise. The results of the trials can be
seen on Tables~\ref{tbl:czeng-czseg}, \ref{tbl:czeng-cztok},
\ref{tbl:czeng-enseg}, \ref{tbl:czeng-entok}, \ref{tbl:brown-seg} and
\ref{tbl:brown-tok}.

\begin{table}
  \begin{center}
    \begin{tabular}{ | l | c | c | c | c | c | }
      \hline
      CzEng - Czech & \multicolumn{4}{ | c | }{Segmentation} \\ \hline
      & Acc. & Prec. & Rec. & F-m. \\ \hline
      Absolute Baseline & 78.08\% & 72.03\% & 98.87\% & 83.34\% \\ \hline
      Simple Tokenizer & 91.23\% & 90.27\% & 94.35\% & 92.27\% \\ \hline
      Groomed Tokenizer & 93.50\% & 91.74\% & 96.97\% & 94.29\% \\
      \hline
    \end{tabular}
  \end{center}
  \caption[Segmentation performance on Czech]{The sentence boundary disambiguiation performance of the various
           methods for tokenizing Czech on the CzEng sample.}
  \label{tbl:czeng-czseg}
\end{table}

\begin{table}
  \begin{center}
    \begin{tabular}{ | l | c | c | c | c | c | }
      \hline
      CzEng - Czech & \multicolumn{4}{ | c | }{Tokenization} \\ \hline
      & Acc. & Prec. & Rec. & F-m. \\ \hline
      Absolute Baseline & 98.92\% & 98.92\% & 100.00\% & 99.45\% \\ \hline
      Simple Tokenizer & 99.17\% & 99.17\% & 100.00\% & 99.58\% \\ \hline
      Groomed Tokenizer & 98.89\% & 98.89\% & 100.00\% & 99.44\% \\
      \hline
    \end{tabular}
  \end{center}
  \caption[Tokenization performance on Czech]{The token boundary disambiguiation performance of the various
           methods for tokenizing Czech on the CzEng sample.}
  \label{tbl:czeng-cztok}
\end{table}

\begin{table}
  \begin{center}
    \begin{tabular}{ | l | c | c | c | c | c | }
      \hline
      CzEng - English & \multicolumn{4}{ | c | }{Segmentation} \\ \hline
      & Acc. & Prec. & Rec. & F-m. \\ \hline
      Absolute Baseline & 82.20\% & 74.14\% & 99.81\% & 85.08\% \\ \hline
      Simple Tokenizer & 93.35\% & 90.41\% & 97.23\% & 93.70\% \\ \hline
      Satz-like System & 92.88\% & 91.92\% & 94.29\% & 93.09\% \\ \hline
      Groomed Tokenizer & 96.34\% & 94.21\% & 98.89\% & 96.49\% \\
      \hline
    \end{tabular}
  \end{center}
  \caption[Segmentation performance on English CzEng]{The sentence boundary disambiguiation performance of the various
           methods for tokenizing English on the CzEng sample.}
  \label{tbl:czeng-enseg}
\end{table}

\begin{table}
  \begin{center}
    \begin{tabular}{ | l | c | c | c | c | c | }
      \hline
      CzEng - English & \multicolumn{4}{ | c | }{Tokenization} \\ \hline
      & Acc. & Prec. & Rec. & F-m. \\ \hline
      Absolute Baseline & 97.14\% & 97.14 & 100.00\% & 98.54\% \\ \hline
      Simple Tokenizer & 99.18\% & 99.42\% & 99.73\% & 99.58\% \\ \hline
      Satz-like System & 99.33\% & 99.32\% & 100.00\% & 99.65\% \\ \hline
      Groomed Tokenizer & 99.03\% & 99.01\% & 100.00\% & 99.50\% \\
      \hline
    \end{tabular}
  \end{center}
  \caption[Tokenization performance on English CzEng]{The token boundary disambiguiation performance of the various
           methods for tokenizing English on the CzEng sample.}
  \label{tbl:czeng-entok}
\end{table}

\begin{table}
  \begin{center}
    \begin{tabular}{ | l | c | c | c | c | c | }
      \hline
      Brown & \multicolumn{4}{ | c | }{Segmentation} \\ \hline
      & Acc. & Prec. & Rec. & F-m. \\ \hline
      Absolute Baseline & 74.95\% & N/A & 0.0\% & N/A \\ \hline
      Simple Tokenizer & 99.33\% & 99.38\% & 98.03\% & 98.70\% \\ \hline
      Satz-like System & 99.50\% & 99.49\% & 98.54\% & 99.01\% \\ \hline
      Groomed Tokenizer & 99.49\% & 99.36\% & 98.59\% & 98.97\% \\
      \hline
    \end{tabular}
  \end{center}
  \caption[Segmentation performance on Brown]{The sentence boundary disambiguiation performance of the various
           methods for tokenizing English on the Brown corpus.}
  \label{tbl:brown-seg}
\end{table}

\begin{table}
  \begin{center}
    \begin{tabular}{ | l | c | c | c | c | c | }
      \hline
      Brown & \multicolumn{4}{ | c | }{Tokenization} \\ \hline
      & Acc. & Prec. & Rec. & F-m. \\ \hline
      Absolute Baseline & 88.27\% & 92.88 & 88.64\% & 90.71\% \\ \hline
      Simple Tokenizer & 98.80\% & 98.55\% & 99.59\% & 99.07\% \\ \hline
      Satz-like System & 98.97\% & 98.75\% & 99.67\% & 99.21\% \\ \hline
      Groomed Tokenizer & 99.75\% & 99.70\% & 99.91\% & 99.80\% \\
      \hline
    \end{tabular}
  \end{center}
  \caption[Tokenization performance on Brown]{The token boundary disambiguiation performance of the various
           methods for tokenizing English on the Brown corpus.}
  \label{tbl:brown-tok}
\end{table}

The Satz-like system which was originally developed on the Brown corpus and
whose part of speech lexica were induced from the tagged training part of the
Brown dataset shows a clear difference in segmentation performance between its
native Brown corpus and the CzEng dataset. This could be perhaps remedied by
improving upon the part of speech lexica. 

The Simple Tokenizer offered quite a lot of performance for a little work.
However, when it comes to segmentation, the difference between the simple
features used by the Simple Tokenizer and the rich features in the Groomed
Tokenizer can show. The Simple Tokenizer also seems to have an affinity for
tokenization. This can be made use of by using one tokenizer with specific
features only for segmentation and then use another for tokenization or vice
versa.

The variations of the methods' performance can be also attributed to the great
difference between the two corpora. The difference can be seen by examining the
precision and recall of the Absolute Baseline. In the CzEng data, a majority of
the potential sentence boundary characters actually are sentence boundaries
whereas in the Brown corpus, 88.27\% of the potential sentence boundaries are
not sentence boundaries. The cause of this difference is most prominently the
fact that in the Brown corpus, a comma can sometimes end a sentence (such as in
poems). This also explains why the tokenizers score so high on the segmentation
of the Brown corpus.

\section{The Speed of the System}
\label{sec:eval-spd}
