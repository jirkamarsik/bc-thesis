\chapter*{Conclusion}
\addcontentsline{toc}{chapter}{Conclusion}

We have presented a data-driven system for tokenizing and segmenting linguistic
data. We have demonstrated the system's versatility by adapting methods based
on different techniques such as morphological analysis, regular expressions and
exception lists. We have demonstrated the system's versatility in being able to
act both as a sentence boundary disambiguator for languages such as English and
Czech and as a word segmenter for languages which cannot rely on whitespace
such as Chinese. We have also pointed to the fact that the program relies only
on multiplatform programs and libraries. While it has not been tested on
Windows or MacOS yet, care was taken at every step to ensure it would be a
smooth transition (ICU can be used instead of libiconv for character code
conversion, CMake is used for building, OS-specific matters are accessed via
Boost...).

The natural next step would be to invent and experiment with new ways and
features for tokenizing and segmenting text. The system offers fast feedback on
the accuracy of the developed systems and is helpful in pointing out positions
which are yet to be covered by rules for inserting decision points. Training
times could be possibly cut down by building a wrapper over the Toolkit for
Advanced Discriminative Modelling.
