\chapter*{Conclusion}
\addcontentsline{toc}{chapter}{Conclusion}

We have presented a data-driven system for tokenizing and segmenting text. We
have demonstrated the system's versatility by combining methods based on
different techniques such as morphological dictionaries, regular expressions
and exception lists. The system proved its universal applicability in being
able to act both as a sentence boundary disambiguator for languages such as
English and Czech and as a word segmenter for languages which do not use
whitespace such as Chinese. We have also pointed to the fact that the program
relies only on multiplatform programs and libraries. While it has not been
tested on Windows or MacOS yet, care was taken at every step to ensure it would
be a smooth transition (ICU can be used instead of libiconv for character code
conversion, CMake is used for building, OS-specific matters are accessed via
Boost only\ldots).

We measured the accuracy, precision, recall and F-measure of the token and
sentence boundary disambiguation. The tests were executed with several very
different tokenization schemes and on several datasets in multiple languages.
We also measured and analyzed the tokenizer's speed and identified the
bottleneck which should serve as an avenue for further optimization.

The natural next step would be to invent and experiment with new ways and
features for tokenizing and segmenting text. The system offers fast feedback on
the accuracy of the user's tokenization schemes and is helpful in pointing out
positions in the text which are yet to be covered by rules for inserting
decision points. Another possible elaboration might be to change the maximum
entropy training backend to the Toolkit for Advanced Discriminative Modelling
or some other alternative.
