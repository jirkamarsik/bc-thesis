\chapter{A Survey of Other Solutions}
\label{chap:survey}

In this chapter we present an overview of existing systems designed to
disambiguate sentence and token boundaries. We examine systems based both on
hand-written rules and systems using machine learning methods such as maximum
entropy models and decision trees. Next, we look at a system that uses part of
speech data to disambiguate sentence boundaries and another system which uses
collocation detection techniques. Finally we describe a state-of-the-art
Chinese word segmenter. For each of these systems, we describe how our
tokenizer can be used to express the same ideas about sentence and token
boundary disambiguation.

\section{RE}
\label{sec:survey-re}

The system \cite{sbd-re} referred to as RE in \cite{sbd-punkt} is an example of
a purely \newterm{rule-based} system. It does not need any training data, but
instead it relies on explicit linguistic knowledge such as lists of
abbreviations and custom regular expressions. The RE system in particular works
by scanning the input text for periods and then inspecting the tokens
surrounding it. If the surrounding tokens do not match a combination of the
user's regular expressions, the period is marked as a sentence boundary.

Our tokenizer also allows the user to define regular expressions against which
neighboring tokens will be checked (not only neighboring tokens, a token at any
distance can be examined, which can be important as we saw in the
introduction). The crucial difference between the RE system and our tokenizer
is that the outcomes of all these regular expression tests are not explicitly
mapped to the disambiguation of the potential boundary by the programmer or the
user. Instead, our system relies on already tokenized data from which it learns
how to combine the outcomes of these regular expression tests into a
tokenization decision.

\section{MxTerminator}
\label{sec:survey-mxterm}

Contrary to RE, MxTerminator \cite{sbd-mxterm} is a \newterm{supervised
machine-learning} system. This means that the tool has to be supplied with
already tokenized data from which the classifier infers the logic behind
tokenization. The classifier in this case is based on maximum entropy models,
the same mathematical foundation on which our system is built.

The MxTerminator scans the text for a list of potential sentence terminators
and presents the classifier with features of the neighboring tokens. The
hard-coded features include the word containing the potential sentence
terminator, the words preceding and following it, the presence of particular
characters in the current word and whether the current word is a honorific or a
corporate designator (e.g.\ Corp.). All of these are easily expressed using
regular expressions and lists of tokens and so it should be quite easy to
produce a system very similar to MxTerminator using a specific configuration.

There is also a more general version of the MxTerminator which does not rely on
precompiled lists of honorifics and other abbreviations. In this version, the
MxTerminator first scans the training data and searches for words containing a
period which does not serve as a sentence terminator. The features passed to
the maximum entropy classifier then consist only of the trigram of words
containing the potential sentence terminator and values describing whether the
individual words belong to the abbreviations induced from training data in the
previous step. With our tokenizer, the user is free to scan the data ahead and
store the induced abbreviations in a file. The tokenizer can then be configured
to use the file as a definition for the induced abbreviation feature.

\section{Riley}
\label{sec:survey-riley}

Riley \cite{sbd-riley} uses a method of classification different from the
MxTerminator. Instead of using a maximum entropy classifier, he builds a
regression tree. The following features are used to disambiguate the period
(let $a$ be the word containing the period in question and $b$ the following
word):

\begin{itemize}
  \item Probability of $a$ occurring at the end of a sentence
  \item Probability of $b$ occurring at the beginning of a sentence
  \item Length of $a$
  \item Length of $b$
  \item Case of $a$
  \item Case of $b$
  \item Any punctuation after the period
  \item Abbreviation class of $a$
\end{itemize}

A training dataset the size of approximately 25 million words was used to
estimate the probabilities of individual words occurring near sentence
boundaries. Thanks to such detailed information, the system was found to
perform notably well.

The first two features used in the regression tree have a natural counterpart
in the maximum entropy model. When the text of a token is being passed to the
maximum entropy classifier during training, it estimates a parameter for each
type of token encountered and each possible outcome (no boundary, token
boundary, sentence boundary). What this parameter does, basically, is that it
describes and retains in the model the probability of encountering a specific
type together with a specific outcome. The equivalent of a probability of a
certain type occurring near the sentence boundary would therefore be the
maximum entropy model's parameter corresponding to the event of that type
appearing together with the sentence boundary outcome.

As for the length features, the maximum entropy toolkit we employed uses a more
general form of a maximum entropy feature which allows for real feature values
instead of only binary values (the only such feature supported by our tokenizer
is the length of a token). The remaining parameters can be described by binary
features defined as regular expressions supplied by the user.

\section{Satz}
\label{sec:survey-satz}

The Satz system \cite{sbd-satz} is another supervised machine-learning system
for sentence boundary disambiguation. It is very unique in that it does not
rely on the superficial characteristics of the shape of the surrounding
tokens. Instead, it passes to the underlying classifier the probability
distribution of parts of speech for every token within the context of the
potential sentence boundary. It is therefore necessary to supply a lexicon
giving the part of speech distribution. If a word is not part of any lexicon,
a series of heuristics try to guess a safe probability distribution given the
word's suffix, case, internal punctuation etc... Thanks to the generalization
provided by the part of speech categories, the system required relatively
small amounts of training data to achieve solid performance. 

In our system, the user is limited to defining binary features and so passing
the probability distributions to the classifier would be out of the question.
However, the authors of the Satz system performed an experiment wherein they
replaced the non-zero probabilities with ones (basically switching from part of
speech probabilities to flags indicating if a given part of speech is
possible). The results of this experiment showed that the resulting system was
trained faster and performed better than the original. Luckily our tokenizer
allows the user to easily define binary features using lists of tokens, i.e.\
lexicons. The only problem would be the heuristics employed with out of
vocabulary words. While all of them can be easily expressed as regular
expressions in our system, there is yet no mechanism to make the tokenizer
treat a part of speech found in a lexicon and a part of speech guessed by a
regular expression heuristic as the same feature which inhibits generalization.

\section{Punkt}
\label{sec:survey-punkt}

The Punkt system \cite{sbd-punkt} is an example of an \newterm{unsupervised
machine-learning} system. This means that Punkt does not need manually
tokenized data for training, it learns from raw untokenized text. The data
Punkt actually uses for training is the text to be tokenized and so besides the
obvious advantage of not having to manually annotate data, the Punkt system
does not have to be afraid of different text domains and genres.

The Punkt system processes the input in multiple stages. In the first stage, it
tries to determine which period-terminated words are abbreviations. A
likelihood ratio is assigned to every such token type in the text describing
the strength of the collocational tie between the type and its
terminating period. A collocation between a type and a following period is
taken as evidence that the type is an abbreviation type. This collocational
score is further penalized by the length of the type and multiplied by the
number of token-internal periods. Finally, a type's abbreviation likelihood is
also exponentially penalized for each instance not followed by a period (so
that common verbs in head-final languages are not picked up as abbreviations).
All types that score higher than a set threshold are considered abbreviations.

After the abbreviations have been determined, every period not following an
abbreviation, an initial or a number is marked as a unambiguous sentence
boundary. Now that some sentence boundaries have already been disambiguated,
the system studies the input again to infer e.g.\ frequent sentence starters,
which are types which form collocations with preceding sentence boundaries. The
rest of the periods are disambiguated in the second stage which examines the
specific tokens and their contexts. Disambiguation may come from the
orthographic heuristic which examines the case of the following token with
respect to how often its type occurred lower-case and upper-case both at the
start of a sentence and mid-sentence. The orthographic heuristic is very robust
and takes into account that many words are written with upper-case first
letters even mid-sentence (such as proper nouns and German nouns). The second
stage also uses the collocational tie between the types surrounding the period
and whether the following type is a frequent sentence starter as evidence
against, resp.\ for, a sentence boundary.

Punkt also demonstrates its language independence by giving remarkable results
on 11 different languages, all without the need to provide annotated data or
perform lengthy parameter tweaking. Emulating Punkt's behaviour using our
tokenizer would be nearly impossible, as it would necessarily lose its
independence on available annotated data and its ability to train from the
input before tokenizing it. On the other hand, our system is able to perform
nontrivial tokenization tasks (such as Chinese word segmentation) on top of the
sentence boundary disambiguation. It is due to the fact that the Punkt system
was designed to solve a very specific problem using linguistic knowledge common
to a lot of languages. Our tokenizer is very general, permitting the user to
tokenize and segment the text in basically any way that is learnable through
binary features expressed with regular expressions or lexicons.

\section{Chinese Word Segmentation}
\label{survey-chinese}

Several attempts at Chinese word segmentation were made using a maximum entropy
classifier. The one developed by Jin Kiat Low, Hwee Tou Ng and Wenyuan Guo in
2005 \cite{seg-chinese-maxent} ranked amongst the highest in the Second
International Chinese Word Segmentation Bakeoff \cite{web-bakeoff}. It
classifies each character as either a single-character word or as a first,
intermediate or last character of a multi-character word. The basic set of
features passed to the classifier is:

\begin{enumerate}
  \item $C_n (n = -2,-1,0,1,2)$
  \item $C_n C_{n+1} (n = -2,-1,0,1)$
  \item $C_{-1} C_1$
  \item $Pu(C_0)$
  \item $T(C_{-2}) T(C_{-1}) T(C_0) T(C_1) T(C_2)$
\end{enumerate}

$C_n$ refers to a character at a position relative to the current one, $Pu$ is
a predicate checking whether a character is a punctuation symbol and $T$ is a
function assigning a character class to characters. The 4 used classes are
numbers, dates (symbols for ``day'', ``month'' and ``year''), English letters
and others. Feature templates 2, 3 and 5 use conjunctions of features, which
means that for all the possible combinations of values, there is a maximum
entropy feature and its corresponding parameter. It was this classifier which
motivated the implementation of conjunction features in our tokenizer.

The Chinese word segmenter relies on even more features derived from searching
the text for words in a lexicon of known words. In our tokenizer, it would be
quite complicated to check for these words due to the fact that every position
is a potential token boundary. This means that the preliminary rough tokens, on
which user-defined predicates are tested, are exactly one character long.
However, this improvement to the Chinese word segmenter is not that crucial. A
bigger issue might be the fact that the Chinese word segmenter trains a
classifier to predict the role of a character in a single or multi character
word, whereas our classifier predicts whether potential token boundaries are
real token boundaries (this means that during training the set of features for
maximum entropy is quite different).
