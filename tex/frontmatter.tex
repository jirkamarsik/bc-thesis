
%%% Titulní strana práce

\pagestyle{empty}
\begin{center}

\large

Charles University in Prague

\medskip

Faculty of Mathematics and Physics

\vfill

{\bf\Large BACHELOR THESIS}

\vfill

\centerline{\mbox{\includegraphics[width=60mm]{img/logo.eps}}}

\vfill
\vspace{5mm}

{\LARGE Jiří Maršík}

\vspace{15mm}

% Název práce přesně podle zadání
{\LARGE\bfseries Fast and Trainable Tokenizer \\ for Natural Languages}

\vfill

% Název katedry nebo ústavu, kde byla práce oficiálně zadána
% (dle Organizační struktury MFF UK)
Institute of Formal and Applied Linguistics

\vfill

\begin{tabular}{rl}

Supervisor of the bachelor thesis: & RNDr. Ondřej Bojar, Ph.D. \\
\noalign{\vspace{2mm}}
Study programme: & Computer Science \\
\noalign{\vspace{2mm}}
Specialization: & General Computer Science \\
\end{tabular}

\vfill

% Zde doplňte rok
Prague 2011

\end{center}

\newpage

%%% Následuje vevázaný list -- kopie podepsaného "Zadání bakalářské práce".
%%% Toto zadání NENÍ součástí elektronické verze práce, nescanovat.

%%% Na tomto místě mohou být napsána případná poděkování (vedoucímu práce,
%%% konzultantovi, tomu, kdo zapůjčil software, literaturu apod.)

\openright

\noindent
Thanks for all the fish!

\newpage

%%% Strana s čestným prohlášením k bakalářské práci

\vglue 0pt plus 1fill

\noindent
I declare that I carried out this bachelor thesis independently, and only with
the cited sources, literature and other professional sources.

\medskip\noindent
I understand that my work relates to the rights and obligations under the Act
No. 121/2000 Coll., the Copyright Act, as amended, in particular the fact that
the Charles University in Prague has the right to conclude a license agreement
on the use of this work as a school work pursuant to Section 60 paragraph 1 of
the Copyright Act.

\vspace{10mm}

\hbox{\hbox to 0.5\hsize{%
In ........ date ............
\hss}\hbox to 0.5\hsize{%
signature
\hss}}

\vspace{20mm}
\newpage

%%% Povinná informační strana bakalářské práce

\vbox to 0.5\vsize{
\setlength\parindent{0mm}
\setlength\parskip{5mm}

\begin{singlespace}
\begin{otherlanguage}{czech}

Název práce:
Rychlý a trénovatelný tokenizér pro přirozené jazyky
% přesně dle zadání

Autor:
Jiří Maršík

Katedra:  % Případně Ústav:
Ústav formální a aplikované lingvistiky
% dle Organizační struktury MFF UK

Vedoucí bakalářské práce:
RNDr. Ondřej Bojar Ph.D., Ústav formální a aplikované lingvistiky
% dle Organizační struktury MFF UK, případně plný název pracoviště mimo MFF UK

Abstrakt:
% abstrakt v rozsahu 80-200 slov; nejedná se však o opis zadání bakalářské
% práce
V této práci představujeme systém pro dezambiguaci hranic mezi tokeny a větami.
Charakteristickým znakem programu je jeho značná konfigurovatelnost a
všestrannost, tokenizér si dokáže poradit např. i s nepřerušovaným čínským
textem. Tokenizér používá klasifikátory založené na modelech s maximální
entropií a jedná se tudíž o systém strojového učení, kterému je nutné předložit
již tokenizovaná data k trénování. Program je doplněn nástrojem pro hlášení
úspěšnosti tokenizace, což pomáhá zejména při rychlém vývoji a ladění
tokenizačního procesu. Systém byl vyvinut pouze za pomoci multiplatformních
knihoven a při vývoji byl kladen důraz zejména na efektivitu a správnost. Velká
část práce se zabývá vlastní implementací tokenizéru a vyhodnocením jeho
úspěšnosti. Dále věnujeme chvíli přehledu jiných tokenizérů a krátkému úvodu do
teorie modelů s maximální entropií.

Klíčová slova:
tokenizace, segmentace, maximální entropie, předzpracování textu
% 3 až 5 klíčových slov

\end{otherlanguage}
\end{singlespace}

\vss}\nobreak\vbox to 0.49\vsize{
\setlength\parindent{0mm}
\setlength\parskip{5mm}

\begin{singlespace}

Title:
Fast and Trainable Tokenizer for Natural Languages
% přesný překlad názvu práce v angličtině

Author:
Jiří Maršík

Department:
Institute of Formal and Applied Linguistics
% dle Organizační struktury MFF UK v angličtině

Supervisor:
RNDr. Ondřej Bojar Ph.D., Institute of Formal and Applied Linguistics
% dle Organizační struktury MFF UK, případně plný název pracoviště
% mimo MFF UK v angličtině

Abstract:
In this thesis, we present a data-driven system for disambiguating token and
sentence boundaries. The implemented system is highly configurable and
versatile to the point its tokenization abilities are able to satisfactorily
segment unbroken Chinese text. The tokenizer relies on maximum entropy
classifiers and requires tokenized and segmented text as training data. The
program is complete with a tool for reporting the performance of the
tokenization which helps rapidly develop and tune the tokenization process. The
system was built with multiplaform libraries only and with emphasis on speed
and completeness. A large part of the thesis focuses on the particular
implementation we developed and its evaluation. The thesis also contains a
survey of other tools for text tokenization and segmentation and a short
introduction to maximum entropy modelling.
% abstrakt v rozsahu 80-200 slov v angličtině; nejedná se však o překlad zadání
% bakalářské práce

Keywords:
% 3 až 5 klíčových slov v angličtině
tokenization, segmentation, maximum entropy, text preprocessing

\end{singlespace}

\vss}

\newpage
