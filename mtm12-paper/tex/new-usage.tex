\section{Usage}
\label{sec:usage}

TrTok is used as a command line application.

\begin{verbatim}
Example:
    trtok train en/satz-like/brown -l data/brown/train.fl -r "|raw|txt|"
\end{verbatim}

Its first argument is the mode of operation, which can be one of
\texttt{train}, \texttt{tokenize}, \texttt{evaluate} or
\texttt{prepare}. The \texttt{train} mode uses manually annotated
files to train a model for the Classifier and save it, while the
\texttt{evaluate} mode uses them to compare the tokenizer's
predictions to the manual tokenization and outputs the comparisons.
The \texttt{tokenize} mode takes the input files and segments them
using the trained model. The \texttt{prepare} mode does the same but
with a dummy model which performs every possible sentence and token
break.

The second argument to TrTok is the tokenization scheme folder. The
tokenization scheme folder contains a set of optional files which
influence the behavior of the tokenizer. Files with the \texttt{.rep}
and \texttt{.listp} extensions define new features in terms of regular
expressions or lists of types, respectively. Files with the
\texttt{.split}, \texttt{.join} and \texttt{.break} extensions contain
pairs of regular expressions which define possible token splits, token
joins and sentence breaks, respectively. The \texttt{features} file
defines which features are to be used from which rough tokens relative
to the possible tokenization operation. The \texttt{maxent.params}
file contains values for tuning the performance of the training
algorithm. The scheme folder also allows a few other configuration
files for convenience. An important thing to note is that the scheme
folders can be nested and that the inner schemes inherit all the files
of the outer scheme, unless they provide their own files of the same
name. This is useful in cases where e.g.\ some features or training
data are applicable to all texts of a language but refinements exist
for various domains or tokenization conventions.

The rest of the parameters are input files and various options for
adjusting the behavior of the tokenizer.

TrTok requires CMake and Quex at runtime, while several multi-platform
libraries are also required at compile time. Further details on the
installation and use of TrTok can be found in the bundled
documentation.
