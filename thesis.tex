%%% Hlavní soubor. Zde se definují základní parametry a odkazuje se na ostatní
%%% části. %%%

%% Verze pro jednostranný tisk:
% Okraje: levý 40mm, pravý 25mm, horní a dolní 25mm
% (ale pozor, LaTeX si sám přidává 1in)
\documentclass[12pt,a4paper]{report}
\setlength\textwidth{145mm}
\setlength\textheight{247mm}
\setlength\oddsidemargin{15mm}
\setlength\evensidemargin{15mm}
\setlength\topmargin{0mm}
\setlength\headsep{0mm}
\setlength\headheight{0mm}
% \openright zařídí, aby následující text začínal na pravé straně knihy
\let\openright=\clearpage

%% Pokud tiskneme oboustranně:
% \documentclass[12pt,a4paper,twoside,openright]{report}
% \setlength\textwidth{145mm}
% \setlength\textheight{247mm}
% \setlength\oddsidemargin{15mm}
% \setlength\evensidemargin{0mm}
% \setlength\topmargin{0mm}
% \setlength\headsep{0mm}
% \setlength\headheight{0mm}
% \let\openright=\cleardoublepage


\usepackage{setspace}
% Rady z KSI veli pouzit line spacing >= 1.5. Nevim, jestli je tim mysleno
% one-and-a-half line spacing v setspace anebo nastaveni baselinestretch.
% onehalfspacing nastavuje baselinestretch na zhruba 1.3 a doublespacing na
% asi 1.6.
\onehalfspacing
%\doublespacing

%% Pokud používáte csLaTeX (doporučeno):
%\usepackage{czech}
%% Pokud nikoliv:
\usepackage[czech,english]{babel}
\selectlanguage{english}
\usepackage[T1]{fontenc}

%% Použité kódování znaků: obvykle latin2, cp1250 nebo utf8:
\usepackage[utf8]{inputenc}

%% Ostatní balíčky
\usepackage{graphicx}
\usepackage{amsthm}
\usepackage{amsmath}
\usepackage{mathtools}

%% Balíček hyperref, kterým jdou vyrábět klikací odkazy v PDF,
%% ale hlavně ho používáme k uložení metadat do PDF (včetně obsahu).
%% POZOR, nezapomeňte vyplnit jméno práce a autora.
\usepackage[ps2pdf,unicode]{hyperref}   % Musí být za všemi ostatními balíčky
\hypersetup{pdftitle=Fast and Trainable Tokenizer for Natural Languages}
\hypersetup{pdfauthor=Jiří Maršík}

%%% Drobné úpravy stylu

% Tato makra přesvědčují mírně ošklivým trikem LaTeX, aby hlavičky kapitol
% sázel příčetněji a nevynechával nad nimi spoustu místa. Směle ignorujte.
\makeatletter
\def\@makechapterhead#1{
  {\parindent \z@ \raggedright \normalfont
   \Huge\bfseries \thechapter. #1
   \par\nobreak
   \vskip 20\p@
}}
\def\@makeschapterhead#1{
  {\parindent \z@ \raggedright \normalfont
   \Huge\bfseries #1
   \par\nobreak
   \vskip 20\p@
}}
\makeatother

% Toto makro definuje kapitolu, která není očíslovaná, ale je uvedena v obsahu.
\def\chapwithtoc#1{
\chapter*{#1}
\addcontentsline{toc}{chapter}{#1}
}

\usepackage{mystyle}

\begin{document}

% Trochu volnější nastavení dělení slov, než je default.
\lefthyphenmin=2
\righthyphenmin=2


%%% Titulní strana práce

\pagestyle{empty}
\begin{center}

\large

Charles University in Prague

\medskip

Faculty of Mathematics and Physics

\vfill

{\bf\Large BACHELOR THESIS}

\vfill

\centerline{\mbox{\includegraphics[width=60mm]{img/logo.eps}}}

\vfill
\vspace{5mm}

{\LARGE Jiří Maršík}

\vspace{15mm}

% Název práce přesně podle zadání
{\LARGE\bfseries Fast and Trainable Tokenizer for Natural Languages}

\vfill

% Název katedry nebo ústavu, kde byla práce oficiálně zadána
% (dle Organizační struktury MFF UK)
Institute of Formal and Applied Linguistics

\vfill

\begin{tabular}{rl}

Supervisor of the bachelor thesis: & RNDr. Ondřej Bojar, Ph.D. \\
\noalign{\vspace{2mm}}
Study programme: & Computer Science \\
\noalign{\vspace{2mm}}
Specialization: & General Computer Science \\
\end{tabular}

\vfill

% Zde doplňte rok
Prague 2011

\end{center}

\newpage

%%% Následuje vevázaný list -- kopie podepsaného "Zadání bakalářské práce".
%%% Toto zadání NENÍ součástí elektronické verze práce, nescanovat.

%%% Na tomto místě mohou být napsána případná poděkování (vedoucímu práce,
%%% konzultantovi, tomu, kdo zapůjčil software, literaturu apod.)

\openright

\noindent
Thanks for all the fish!

\newpage

%%% Strana s čestným prohlášením k bakalářské práci

\vglue 0pt plus 1fill

\noindent
I declare that I carried out this bachelor thesis independently, and only with
the cited sources, literature and other professional sources.

\medskip\noindent
I understand that my work relates to the rights and obligations under the Act
No. 121/2000 Coll., the Copyright Act, as amended, in particular the fact that
the Charles University in Prague has the right to conclude a license agreement
on the use of this work as a school work pursuant to Section 60 paragraph 1 of
the Copyright Act.

\vspace{10mm}

\hbox{\hbox to 0.5\hsize{%
In ........ date ............
\hss}\hbox to 0.5\hsize{%
signature
\hss}}

\vspace{20mm}
\newpage

%%% Povinná informační strana bakalářské práce

\vbox to 0.5\vsize{
\setlength\parindent{0mm}
\setlength\parskip{5mm}

\begin{otherlanguage}{czech}

Název práce:
Rychlý a trénovatelný tokenizér pro přirozené jazyky
% přesně dle zadání

Autor:
Jiří Maršík

Katedra:  % Případně Ústav:
Ústav formální a aplikované lingvistiky
% dle Organizační struktury MFF UK

Vedoucí bakalářské práce:
RNDr. Ondřej Bojar Ph.D., Ústav formální a aplikované lingvistiky
% dle Organizační struktury MFF UK, případně plný název pracoviště mimo MFF UK

Abstrakt:
% abstrakt v rozsahu 80-200 slov; nejedná se však o opis zadání bakalářské
% práce

Klíčová slova:
% 3 až 5 klíčových slov

\end{otherlanguage}

\vss}\nobreak\vbox to 0.49\vsize{
\setlength\parindent{0mm}
\setlength\parskip{5mm}

Title:
Fast and Trainable Tokenizer for Natural Languages
% přesný překlad názvu práce v angličtině

Author:
Jiří Maršík

Department:
Institute of Formal and Applied Linguistics
% dle Organizační struktury MFF UK v angličtině

Supervisor:
RNDr. Ondřej Bojar Ph.D., Institute of Formal and Applied Linguistics
% dle Organizační struktury MFF UK, případně plný název pracoviště
% mimo MFF UK v angličtině

Abstract:
% abstrakt v rozsahu 80-200 slov v angličtině; nejedná se však o překlad
% zadání bakalářské práce

Keywords:
% 3 až 5 klíčových slov v angličtině

\vss}

\newpage


%%% Strana s automaticky generovaným obsahem bakalářské práce. U matematických
%%% prací je přípustné, aby seznam tabulek a zkratek, existují-li, byl umístěn
%%% na začátku práce, místo na jejím konci.

\openright
\pagestyle{plain}
\setcounter{page}{1}
\begin{singlespace}
  \tableofcontents
\end{singlespace}

%%% Jednotlivé kapitoly práce jsou pro přehlednost uloženy v samostatných
%%% souborech
\chapter*{Introduction}
\addcontentsline{toc}{chapter}{Introduction}

The goal of this thesis was to provide a fast implementation of a system for
disambiguating token and sentence boundaries and to evaluate the
implementation both in terms of its accuracy and its speed.

Token and sentence boundary disambiguation may seem trivial at first, and it
usually is, but in some occasions it might turn out to be quite complex.
Consider the following cases:

\begin{exe}
  \ex{\label{ex:long-context-a}
      On Friday, the 22\textsuperscript{nd}, at around 2 a.m.\ Dr.~T.~Adams
      finished the preliminary examination.}
  \ex{\label{ex:long-context-b}
      The field tests were to begin on Friday, the 22\textsuperscript{nd}, at
      around 2 a.m. Dr.~T.~Adams finished the preliminary examination the night
      before.}
  \ex{"314 159.26\$, about half of the yearly budget, was spent on office
      redecoration!", protested the disgruntled employee of Vanity, S.p.A\@.}
\end{exe}

Even as I was typesetting these examples in \LaTeX{}, I had to explicitly mark
some of the periods in the above examples as not being sentence boundaries, as
\LaTeX{} likes to instert slighly larger spaces after sentence terminators (so
called French spacing). The heuristic used by \LaTeX{} is very simple: if a
word-final potential sentence terminator (a period, a question mark or an
exclamation mark) follows a capital letter, then it is most likely a part of
an abbreviation (or an initial) and so it does not mark the end of a
sentence\footnote{TODO: viz.mail} \cite{web-latex}.

Such a simple system runs into problems in the examples given above, as we
can see that abbreviations do not necessarily end with capital letters and on
top of that a period may serve both as part of an abbreviation and a sentence
terminator. Examples \ref{ex:long-context-a} and \ref{ex:long-context-b} also
show us that the context needed to disambiguate the sentence boundary may be
quite far from the boundary in question.

While getting the size of a space correctly down to the last millimeter is
certainly a noble goal, there are also some important uses for a more reliable
segmenter and tokenizer. When text is being processed and parsed by automatic
tools, a common first step is to divide the text into tokens and sentences. A
lot of the tools that then work with these tokens assume they are correct and
try to analyze them further. As a lot of these tools are getting more and more
accurate, it is important we step up the quality of the tokenization process,
so that the system's quality is not determined by something as basic as
tokenization and segmentation of input. 

In the last 20 years, the problem started getting some recognition and several
systems were demonstrated. This thesis does not aim to create a new system for
tokenization. This work is based on an already existing tokenizer implemented
by Ondřej Bojar during the construction of the UMC 0.1 Czech-Russian-English
Multilingual Corpus \cite{maxent-original-paper,maxent-original}.

A key feature of the original tokenizer is its strict segregation of
language-dependent knowledge into configurable files. The new implementation
expands on this idea and assumes next to nothing about the language being
processed except that the sentence and token boundaries are disambiguated by a
limited context window described by binary predicates expressed as regular
expressions. The tokenizer thus offers a great deal of customizability and a
lot of effort has been put into ensuring that the tokenizer will behave as
expected and that the behaviour is easy to understand without diverging too
much from the original.

Performance, being the motivation behind the current implementation, was also
important. Both the original and the new tokenizer rely on a C++ toolkit which
handles the mechanics of machine learning \cite{maxent-toolkit}. However, the
original implementation, being written in Perl, had to access the functionality
through a command-line interface passing data through files. The new
implementation will have the benefits of using the C++ API directly. Where the
old implementation used regular expressions to partition the input and detect
potential token and sentence boundaries, the new implementation uses a lexical
analyzer generator \cite{web-quex} to generate fast C++ code, compile it and
load it at runtime. The new implementation also benefits from the multiple CPUs
found on modern computers and uses a high-level parallelism library
\cite{web-tbb} to perform the various time-consuming tasks of tokenization in
parallel.

In Chapter~\ref{chap:survey}, we will look at other systems which tried to
tackle the problem and compare them to our tokenizer. In
Chapter~\ref{chap:maxent}, a brief overview of the maximum entropy method of
machine learning will be given. Chapter~\ref{chap:impl} will familiarize us
with the implementation of the tokenizer. Finally, in Chapter~\ref{chap:eval},
we evaluate the speed and accuracy of the tokenizer on several datasets.

\chapter{A Survey of Other Solutions}
\label{chap:survey}

In this chapter we present an overview of existing systems designed to
disambiguate sentence and token boundaries. We examine systems based both on
hand-written rules and systems using machine learning methods such as maximum
entropy models and decision trees. Next, we look at a system that uses part of
speech data to disambiguate sentence boundaries and another system which uses
collocation detection techniques. Finally we describe a state-of-the-art
Chinese word segmenter. For each of these systems, we describe how our
tokenizer can be used to express the same ideas about sentence and token
boundary disambiguation.

\section{RE}
\label{sec:survey-re}

The system \cite{sbd-re} referred to as RE in \cite{sbd-punkt} is an example of
a purely \newterm{rule-based} system. It does not need any training data, but
instead it relies on explicit linguistic knowledge such as lists of
abbreviations and custom regular expressions. The RE system in particular works
by scanning the input text for periods and then inspecting the tokens
surrounding it. If the surrounding tokens do not match a combination of the
user's regular expressions, the period is marked as a sentence boundary.

Our tokenizer also allows the user to define regular expressions against which
neighboring tokens will be checked (not only neighboring tokens, a token at any
distance can be examined, which can be important as we saw in the
introduction). The crucial difference between the RE system and our tokenizer
is that the outcomes of all these regular expression tests are not explicitly
mapped to the disambiguation of the potential boundary by the programmer or the
user. Instead, our system relies on already tokenized data from which it learns
how to combine the outcomes of these regular expression tests into a
tokenization decision.

\section{MxTerminator}
\label{sec:survey-mxterm}

Contrary to RE, MxTerminator \cite{sbd-mxterm} is a \newterm{supervised
machine-learning} system. This means that the tool has to be supplied with
already tokenized data from which the classifier infers the logic behind
tokenization. The classifier in this case is based on maximum entropy models,
the same mathematical foundation on which our system is built.

The MxTerminator scans the text for a list of potential sentence terminators
and presents the classifier with features of the neighboring tokens. The
hard-coded features include the word containing the potential sentence
terminator, the words preceding and following it, the presence of particular
characters in the current word and whether the current word is a honorific or a
corporate designator (e.g.\ Corp.). All of these are easily expressed using
regular expressions and lists of tokens and so it should be quite easy to
produce a system very similar to MxTerminator using a specific configuration.

There is also a more general version of the MxTerminator which does not rely on
precompiled lists of honorifics and other abbreviations. In this version, the
MxTerminator first scans the training data and searches for words containing a
period which does not serve as a sentence terminator. The features passed to
the maximum entropy classifier then consist only of the trigram of words
containing the potential sentence terminator and values describing whether the
individual words belong to the abbreviations induced from training data in the
previous step. With our tokenizer, the user is free to scan the data ahead and
store the induced abbreviations in a file. The tokenizer can then be configured
to use the file as a definition for the induced abbreviation feature.

\section{Riley}
\label{sec:survey-riley}

Riley \cite{sbd-riley} uses a method of classification different from the
MxTerminator. Instead of using a maximum entropy classifier, he builds a
regression tree. The following features are used to disambiguate the period
(let $a$ be the word containing the period in question and $b$ the following
word):

\begin{itemize}
  \item Probability of $a$ occuring at the end of a sentence
  \item Probability of $b$ occuring at the beginning of a sentence
  \item Length of $a$
  \item Length of $b$
  \item Case of $a$
  \item Case of $b$
  \item Any punctuation after the period
  \item Abbreviation class of $a$
\end{itemize}

A training dataset the size of approximately 25 million words was used to
estimate the probabilities of individual words occuring near sentence
boundaries. Thanks to such detailed information, the system was found to
perform notably well.

The first two features used in the regression tree have a natural counterpart
in the maximum entropy model. When the text of a token is being passed to the
maximum entropy classifier during training, it estimates a parameter for each
type of token encountered and each possible outcome (no boundary, token
boundary, sentence boundary). What this parameter does, basically, is that it
describes and retains in the model the probability of encountering a specific
type together with a specific outcome. The equivalent of a probability of a
certain type occuring near the sentence boundary would therefore be the
maximum entropy model's parameter corresponding to the event of that type
appearing together with the sentence boundary outcome.

As for the length features, the maximum entropy toolkit we employed uses a more
general form of a maximum entropy feature which allows for real feature values
instead of only binary values (the only such feature supported by our tokenizer
is the length of a token). The remaining parameters can be described by binary
features defined as regular expressions supplied by the user.

\section{Satz}
\label{sec:survey-satz}

The Satz system \cite{sbd-satz} is another supervised machine-learning system
for sentence boundary disambiguation. It is very unique in that it does not
rely on the superficial characteristics of the shape of the surrounding
tokens. Instead, it passes to the underlying classifier the probability
distribution of parts of speech for every token within the context of the
potential sentence boundary. It is therefore necessary to supply a lexicon
giving the part of speech distribution. If a word is not part of any lexicon,
a series of heuristics try to guess a safe probability distribution given the
word's suffix, case, internal punctuation etc... Thanks to the generalization
provided by the part of speech categories, the system required relatively
small amounts of training data to acheive solid performance. 

In our system, the user is limited to defining binary features and so passing
the probability distributions to the classifier would be out of the question.
However, the authors of the Satz system performed an experiment wherein they
replaced the non-zero probabilities with ones (basically switching from part of
speech probabilities to flags indicating if a given part of speech is
possible). The results of this experiment showed that the resulting system was
trained faster and performed better than the original. Luckily our tokenizer
allows the user to easily define binary features using lists of tokens, i.e.\
lexicons. The only problem would be the heuristics employed with out of
vocabulary words. While all of them can be easily expressed as regular
expressions in our system, there is yet no mechanism to make the tokenizer
treat a part of speech found in a lexicon and a part of speech guessed by a
heuristic as the same feature which inhibits the generalization.

\section{Punkt}
\label{sec:survey-punkt}

The Punkt system \cite{sbd-punkt} is an example of an \newterm{unsupervised
machine-learning} system. This means that Punkt does not need manually
tokenized data for training, it learns from raw untokenized text. The data
Punkt actually uses for training is the text to be tokenized and so besides the
obvious advantage of not having to manually annotate data, the Punkt system
does not have to be afraid of different text domains and genres.

The Punkt system processes the input in multiple stages. In the first stage, it
tries to determine which period-terminated words are abbreviations. A
likelihood ratio is assigned to every such token type in the text describing
the strength of the collocational tie between the type and its
terminating period. A collocation between a type and a following period is
taken as evidence that the type is an abbreviation type. This collocational
score is further penalized by the length of the type and multiplied by the
number of token-internal periods. Finally, a type's abbreviation likelihood is
also exponentially penalized for each instance not followed by a period (so
that common verbs in head-final languages are not picked up as abbreviations).
All types that score higher than a set threshold are considered abbreviations.

After the abbreviations have been determined, every period not following an
abbreviation, an initial or a number is marked as a nonambiguous sentence
boundary. Now that some sentence boundaries have already been disambiguated,
the system studies the input again to infer e.g.\ frequent sentence starters,
which are types which form collocations with preceding sentence boundaries. The
rest of the periods are disambiguated in the second stage which examines the
specific tokens and their contexts. Disambiguation may come from the
orthographic heuristic which examines the case of the following token with
respect to how often its type occured lower-case and upper-case both at the
start of a sentence and mid-sentence. The orthographic heuristic is very robust
and takes into account that many words are written with upper-case first
letters even mid-sentence (such as proper nouns and German nouns). The second
stage also uses the collocational tie between the types surrounding the period
and whether the following type is a frequent sentence starter as evidence
against, resp.\ for, a sentence boundary.

Punkt also demonstrates its language independence by giving remarkable results
on 11 different languages, all without the need to provide annotated data or
perform lengthy parameter tweaking. Emulating Punkt's behaviour using our
tokenizer would be nearly impossible, as it would necessarily lose its
independence on available annotated data and its ability to train from the
input before tokenizing it. On the other hand, our system is able to perform
nontrivial tokenization tasks (such as Chinese word segmentation) on top of the
sentence boundary disambiguation. It is due to the fact that the Punkt system
was designed to solve a very specific problem using linguistic knowledge common
to a lot of languages. Our tokenizer is very general, permitting the user to
tokenize and segment the text in basically any way that is learnable through
binary features expressed with regular expressions or lexicons.

\section{Chinese Word Segmentation}
\label{survey-chinese}

Several attempts at Chinese word segmentation were made using a maximum entropy
classifier. The one developed by Jin Kiat Low, Hwee Tou Ng and Wenyuan Guo in
2005 \cite{seg-chinese-maxent} ranked amongst the highest in the Second
International Chinese Word Segmentation Bakeoff. It classifies each character
as a single-character word or as a first, intermediate or last character of a
multi-character word. The basic set of features passed to the classifier is:

\begin{enumerate}
  \item $C_n (n = -2,-1,0,1,2)$
  \item $C_n C_{n+1} (n = -2,-1,0,1)$
  \item $C_{-1} C_1$
  \item $Pu(C_0)$
  \item $T(C_{-2}) T(C_{-1}) T(C_0) T(C_1) T(C_2)$
\end{enumerate}

$C_n$ refers to a character at a position relative to the current one, $Pu$ is
a predicate checking whether a character is a punctuation symbol and $T$ is a
function assigning a character class to characters. The 4 used classes are
numbers, dates (symbols for ``day'', ``month'' and ``year''), English letters
and others. Feature templates 2, 3 and 5 use conjunctions of features, which
means that for all the possible combinations of values, there is a maximum
entropy feature and its corresponding parameter. It was this classifier which
motivated the implementation of conjunction features in our tokenizer.

The Chinese word segmenter relies on even more features derived from searching
the text for words in a lexicon of known words. In our tokenizer, it would be
quite complicated to check for these words due to the fact that every position
is a potential token boundary. This means that the preliminary rough tokens, on
which user-defined predicates are tested, are exactly one character long.
However, this improvement to the Chinese word segmenter is not that crucial. A
bigger issue might be the fact that the Chinese word segmenter trains a
classifier to predict the role of a character in a single or multi character
word, whereas our classifier predicts whether potential token boundaries are
real token boundaries (this means that during training the set of features for
maximum entropy is quite different).

\chapter{Maximum Entropy Models}
\label{chap:maxent}

We want to construct a probabilistic model which gives us a probability
$p(a,b)$ of an outcome\footnote{The terminology used in computational
linguistics often clashes with the one used in probability theory. What is in
probability theory usually known as an outcome is here referred to as an
\newterm{event}. These events are pairs of \newterm{contexts} and
\newterm{outcomes}, where the context is the data we have available when we
want a prediction and the outcome is what we want to predict.} $a$ occuring
with context $b$. We want this model to be very close to the observed training
data, meaning that the data's probability given our model $p$ is high.

However, we do not want the maximum likelihood model because we
are aware that the observed data does not cover all the possible situations.
Instead, we want a model that shares a lot of properties with the observed
data. We express these properties as binary functions on the space of events
$E$ and we call these functions \newterm{features}\footnote{The term features
is also commonly used in machine learning to denote a part of the context. When
it will be important to differentiate these two meaning in other parts of the
work, the term \newterm{maximum entropy features} will be used to refer to the
features defined here.}. In most implementations, including ours, these binary
features are restricted to the following form

\begin{equation}
\label{eq:common-feature}
f(a,b) =
\begin{cases}
  1 & \text{if } a=o \text{ and } p(b) \\
  0 & \text{else}
\end{cases}
\end{equation}

where $o$ is an outcome and $p$ is a context predicate. We want the constructed
model $p$ to share the expected values of these feature functions with the
empirical model $\bar{p}$. What this means is that the probability of $f(a,b)$
being 1 is the same in both models.

Let us say we have chosen several such features we want retained in our model,
now we need to select some model from the set of complying models. This is the
point where the maximum entropy entropy principle comes into play. The basic
idea of the maximum entropy principle was nicely hinted at by Laplace:

\begin{quote}
When one has no information to distinguish between the probability of two
events, the best strategy is to consider them equally likely.
\end{quote}

We would like to have a distribution which conforms to the requirements imposed
by the features but is otherwise unbiased, it is as close to uniform as
possible without violating the features' requirements. A standard measure of
the uniformity of a distribution is entropy

\[
H(p) = -\sum_{x \in E} p(x) \log p(x)
\]

We would like to find a distribution which adheres to the features' constraints
and maximizes the maximum entropy. It can be shown (see e.g.\ Ratnaparkhi) that
the such a distribution is of the following form

\begin{equation}
\label{eq:exp-model}
p(x) = \pi \prod_{j=1}^k \alpha_j^{f_j(x)}
\end{equation}

where $f_j(x)$ for $j \in \{1,\dotsc,k\}$ are the features we want to retain
and $0 < \alpha_j < \infty$. More interestingly, the maximum entropy model
adhering to the features' constraints is equal to the maximum likelihood model
having the shape in \ref{eq:exp-model} (we call them \newterm{exponential
models}). 

Given the set of features we want to retain in our model, we can now employ an
unrestricted optimization algorithm to find the parameters of the exponential
model which maximizes the likelihood of the training data.

Once we wrap our minds around the definition of an exponential model and the
possible features from \ref{eq:common-feature}, we can see that when predicting
an outcome given a context, each predicate which holds for the context votes
for each outcome by multiplying its probability. The probability is multiplied
by the parameter of the exponential model corresponding to the feature which
is constructed from the outcome and predicate in question. The probability is
either increased or decreased depending on how often was that predicate
encountered with the same outcome in the training data.

In practice, the features (in the machine learning sense of the word) being
passed to the maximum entropy classifier are the predicates which combine with
the outcomes to form the maximum entropy features.

\chapter{Implementation}
\label{chap:impl}

\begin{figure}[ht]
  \includegraphics[width=\textwidth]{img/all-parts.eps}
  \caption{Data flow in the entire system}
  \label{fig:all-parts}
\end{figure}



\chapter{Evaluation}
\label{chap:eval}



\section{The Accuracy of the System}
\label{sec:eval-acc}

\subsection{Chinese Word Segmentation}
\label{ssec:eval-acc-chinese}

Tokenizing latin-script languages is not very hard. We can usually get by well
enough by splitting the text at whitespaces and at boundaries between different
classes of symbols. Sometimes, we might want to be more specific and try to
tokenize English contractions as separate words. However, these problems are
quite easy to solve when compared to the task of tokenizing Chinese text. The
absence of any spaces between words forbids the use of any simple heuristic and
linguistically empowered methods must be used.

We took inspiration from the system for Chinese word segmentation presented in
Section~\ref{survey-chinese} which is also based on maximum entropy models. The
basic features used in that system were ported to our formalism. The biggest
difference between the systems was the fact that the original Chinese tokenizer
classified individual characters as being single words or the beginnings,
middles or ends of a multi-character word. However, the classifier used in
our system is binary and it decides for each character boundary whether it
forms a token boundary or not.

We were able to obtain the same data on which the original tokenizer was
developed, which happen to be the training data for the Second International
Chinese Word Segmentation Bakeoff. The bakeoff was a competition challenging
computational linguists to develop word segmentation systems for Chinese using
the supplied data for training. The provided data consists of 4 datasets
provided by Academia Sinica, City University of Hong Kong, Peking University
and Microsoft Research. Each of these datasets adopts slightly different
tokenization standards and so we train and test our tokenizer on the datasets
individually. Each dataset comes with a training part and a testing part. We
strictly used only the training part when developing our tokenizer and used the
testing part only at the end to evaluate our results. The only thing we knew
about the testing data in advance was its size which helped us choose a
reasonable size for our heldout data.

First off, we split our training data into a development part and a heldout
part. We chose the size of the heldout data to be roughly as big as the testing
data so we could trust our system's performance on it to be representative of
our system's true accuracy. The sizes of the partitioned datasets can be seen
in Table~\ref{tbl:bakeoff-sizes}.

\begin{table}
  \begin{center}
    \begin{tabular}{ | l | r | r | r | }
      \hline
      & \multicolumn{2}{ | c | }{Training data} & Testing data \\ \hline
      & Development data & Heldout data & Testing data \\ \hline
      Academia Sinica & 39686533 & 1057344 & 942571 \\ \hline
      City University & 8283540 & 266129 & 240767 \\ \hline
      Peking University & 7008808 & 719430 & 718331 \\ \hline
      Microsoft Research & 16100177 & 791333 & 766786 \\
      \hline
    \end{tabular}
  \end{center}
  \caption[Bakeoff dataset sizes]{The sizes of the individual parts of the bakeoff datasets in bytes.}
  \label{tbl:bakeoff-sizes}
\end{table}

We set the event cutoff to 2 as in [], so we retain a lot of the encountered
bigrams but still keep the number of parameters manageable. We then
experimented with training the tokenizer and testing it on the heldout data.
Depending on how much we constrained training time, the tokenizer could either
be undertrained or overfitted. The heldout data served as an independent
indicator telling us how close we are to the ideal balance between a detailed
and a general model. Experimentation led us to restrain the number of training
iterations to the values seen in Table~\ref{tbl:bakeoff-iters}. We can see that
the suitable number of iterations spent training is nearly a linear function of
the dataset size. This stems from the fact that a larger dataset means more
bigrams and unigrams and thus more parameters to estimate.

\begin{table}
  \begin{center}
    \begin{tabular}{ | l | r | }
      \hline
      & Number of iterations \\ \hline
      Academia Sinica & 1000 \\ \hline
      City University & 175 \\ \hline
      Peking University & 150 \\ \hline
      Microsoft Research & 400 \\
      \hline
    \end{tabular}
  \end{center}
  \caption[Recommended iterations for Chinese segmentation]{A suitable number of iterations for training on a given dataset in
           the bakeoff data.}
  \label{tbl:bakeoff-iters}
\end{table}

After we established the training parameters, we trained the system on the
entier training data and checked its performance on the gold testing data. The
performance of the development system on the heldout data and of the final
system on the testing data can be seen in Tables \ref{tbl:bakeoff-devel} and
\ref{tbl:bakeoff-final}.

\begin{table}
  \begin{center}
    \begin{tabular}{ | l | r | r | r | r | }
      \hline
      & Accuracy & Precision & Recall & F-measure \\ \hline
      Academia Sinica & 95.55\% & 96.38\% & 95.78\% & 96.08\% \\ \hline
      City University & 91.58\% & 94.28\% & 91.67\% & 92.95\% \\ \hline
      Peking University & 92.70\% & 94.20\% & 93.70\% & 93.95\% \\ \hline
      Microsoft Research & 93.19\% & 95.71\% & 92.79\% & 94.22\% \\
      \hline
    \end{tabular}
  \end{center}
  \caption[Development performance of Chinese segmenter]{The performance of the system trained on the development data when
           tokenizing the heldout data.}
  \label{tbl:bakeoff-devel}
\end{table}

\begin{table}
  \begin{center}
    \begin{tabular}{ | l | r | r | r | r | }
      \hline
      & Accuracy & Precision & Recall & F-measure \\ \hline
      Academia Sinica & 93.82\% & 94.63\% & 94.92\% & 94.77\% \\ \hline
      City University & 91.75\% & 94.38\% & 91.62\% & 92.98\% \\ \hline
      Peking University & 91.94\% & 94.94\% & 91.43\% & 93.15\% \\ \hline
      Microsoft Research & 94.34\% & 96.19\% & 93.79\% & 94.98\% \\
      \hline
    \end{tabular}
  \end{center}
  \caption[Final performance of Chinese segmenter]{The performance of the system trained on the entire training data
           when tokenizing the gold testing data.}
  \label{tbl:bakeoff-final}
\end{table}

While the resulting adapted system does not perform as well as the original
word segmenter by Low, Ng and Guo, it achieves a median performance compared to
the performance of the other bakeoff submissions. The result is quite pleasing,
given that the all we needed was to write the feature definitions into a few
files and toy with some training parameters.

\subsection{Tokenization of Czech and English}
\label{ssec:eval-acc-eng}

For evaluating the accuracy of tokenizing Czech and English text, four
different methods were implemented. The Absolute Baseline relies on no other
piece of information than the current decision point and the whitespace
following it to classifiy boundaries. It is there to show the minimum possible
line every tokenizer should pass. The Simple Tokenizer checks for capital
letters before the period (initials) and after the period (start of a new
sentence). It represents the often too simple approach to tokenization. The
English-only Satz-like system uses only part of speech data about the
surrounding tokens to predict a boundary. Finally, the Groomed Tokenizer is the
tokenization scheme used in the original reference implementation, which has
been supplied with lists of abbreviations and lots of useful regular
expressions.

All systems were tested both on a sample of data from CzEng and, in the case of
the English tests, also on the Brown corpus. All datasets were divided into
equally large development, heldout and testing sets to be used as in
Section~\label{ssec:eval-acc-chinese}. As for the part of speech data of the
Satz-like system, lexicons for each part of speech were extracted from the
training section of the Brown corpus for the Brown corpus exercise and from the
entire Brown corpus for the CzEng exercise. The results of the trials can be
seen on Tables~\ref{tbl:czeng-czseg}, \ref{tbl:czeng-cztok},
\ref{tbl:czeng-enseg}, \ref{tbl:czeng-entok}, \ref{tbl:brown-seg} and
\ref{tbl:brown-tok}.

\begin{table}
  \begin{center}
    \begin{tabular}{ | l | c | c | c | c | c | }
      \hline
      CzEng - Czech & \multicolumn{4}{ | c | }{Segmentation} \\ \hline
      & Acc. & Prec. & Rec. & F-m. \\ \hline
      Absolute Baseline & 78.08\% & 72.03\% & 98.87\% & 83.34\% \\ \hline
      Simple Tokenizer & 91.23\% & 90.27\% & 94.35\% & 92.27\% \\ \hline
      Groomed Tokenizer & 93.50\% & 91.74\% & 96.97\% & 94.29\% \\
      \hline
    \end{tabular}
  \end{center}
  \caption[Segmentation performance on Czech]{The sentence boundary disambiguiation performance of the various
           methods for tokenizing Czech on the CzEng sample.}
  \label{tbl:czeng-czseg}
\end{table}

\begin{table}
  \begin{center}
    \begin{tabular}{ | l | c | c | c | c | c | }
      \hline
      CzEng - Czech & \multicolumn{4}{ | c | }{Tokenization} \\ \hline
      & Acc. & Prec. & Rec. & F-m. \\ \hline
      Absolute Baseline & 98.92\% & 98.92\% & 100.00\% & 99.45\% \\ \hline
      Simple Tokenizer & 99.17\% & 99.17\% & 100.00\% & 99.58\% \\ \hline
      Groomed Tokenizer & 98.89\% & 98.89\% & 100.00\% & 99.44\% \\
      \hline
    \end{tabular}
  \end{center}
  \caption[Tokenization performance on Czech]{The token boundary disambiguiation performance of the various
           methods for tokenizing Czech on the CzEng sample.}
  \label{tbl:czeng-cztok}
\end{table}

\begin{table}
  \begin{center}
    \begin{tabular}{ | l | c | c | c | c | c | }
      \hline
      CzEng - English & \multicolumn{4}{ | c | }{Segmentation} \\ \hline
      & Acc. & Prec. & Rec. & F-m. \\ \hline
      Absolute Baseline & 82.20\% & 74.14\% & 99.81\% & 85.08\% \\ \hline
      Simple Tokenizer & 93.35\% & 90.41\% & 97.23\% & 93.70\% \\ \hline
      Satz-like System & 92.88\% & 91.92\% & 94.29\% & 93.09\% \\ \hline
      Groomed Tokenizer & 96.34\% & 94.21\% & 98.89\% & 96.49\% \\
      \hline
    \end{tabular}
  \end{center}
  \caption[Segmentation performance on English CzEng]{The sentence boundary disambiguiation performance of the various
           methods for tokenizing English on the CzEng sample.}
  \label{tbl:czeng-enseg}
\end{table}

\begin{table}
  \begin{center}
    \begin{tabular}{ | l | c | c | c | c | c | }
      \hline
      CzEng - English & \multicolumn{4}{ | c | }{Tokenization} \\ \hline
      & Acc. & Prec. & Rec. & F-m. \\ \hline
      Absolute Baseline & 97.14\% & 97.14 & 100.00\% & 98.54\% \\ \hline
      Simple Tokenizer & 99.18\% & 99.42\% & 99.73\% & 99.58\% \\ \hline
      Satz-like System & 99.33\% & 99.32\% & 100.00\% & 99.65\% \\ \hline
      Groomed Tokenizer & 99.03\% & 99.01\% & 100.00\% & 99.50\% \\
      \hline
    \end{tabular}
  \end{center}
  \caption[Tokenization performance on English CzEng]{The token boundary disambiguiation performance of the various
           methods for tokenizing English on the CzEng sample.}
  \label{tbl:czeng-entok}
\end{table}

\begin{table}
  \begin{center}
    \begin{tabular}{ | l | c | c | c | c | c | }
      \hline
      Brown & \multicolumn{4}{ | c | }{Segmentation} \\ \hline
      & Acc. & Prec. & Rec. & F-m. \\ \hline
      Absolute Baseline & 74.95\% & N/A & 0.0\% & N/A \\ \hline
      Simple Tokenizer & 99.33\% & 99.38\% & 98.03\% & 98.70\% \\ \hline
      Satz-like System & 99.50\% & 99.49\% & 98.54\% & 99.01\% \\ \hline
      Groomed Tokenizer & 99.49\% & 99.36\% & 98.59\% & 98.97\% \\
      \hline
    \end{tabular}
  \end{center}
  \caption[Segmentation performance on Brown]{The sentence boundary disambiguiation performance of the various
           methods for tokenizing English on the Brown corpus.}
  \label{tbl:brown-seg}
\end{table}

\begin{table}
  \begin{center}
    \begin{tabular}{ | l | c | c | c | c | c | }
      \hline
      Brown & \multicolumn{4}{ | c | }{Tokenization} \\ \hline
      & Acc. & Prec. & Rec. & F-m. \\ \hline
      Absolute Baseline & 88.27\% & 92.88 & 88.64\% & 90.71\% \\ \hline
      Simple Tokenizer & 98.80\% & 98.55\% & 99.59\% & 99.07\% \\ \hline
      Satz-like System & 98.97\% & 98.75\% & 99.67\% & 99.21\% \\ \hline
      Groomed Tokenizer & 99.75\% & 99.70\% & 99.91\% & 99.80\% \\
      \hline
    \end{tabular}
  \end{center}
  \caption[Tokenization performance on Brown]{The token boundary disambiguiation performance of the various
           methods for tokenizing English on the Brown corpus.}
  \label{tbl:brown-tok}
\end{table}

The Satz-like system which was originally developed on the Brown corpus and
whose part of speech lexica were induced from the tagged training part of the
Brown dataset shows a clear difference in segmentation performance between its
native Brown corpus and the CzEng dataset. This could be perhaps remedied by
improving upon the part of speech lexica. 

The Simple Tokenizer offered quite a lot of performance for a little work.
However, when it comes to segmentation, the difference between the simple
features used by the Simple Tokenizer and the rich features in the Groomed
Tokenizer can show. The Simple Tokenizer also seems to have an affinity for
tokenization. This can be made use of by using one tokenizer with specific
features only for segmentation and then use another for tokenization or vice
versa.

The variations of the methods' performance can be also attributed to the great
difference between the two corpora. The difference can be seen by examining the
precision and recall of the Absolute Baseline. In the CzEng data, a majority of
the potential sentence boundary characters actually are sentence boundaries
whereas in the Brown corpus, 88.27\% of the potential sentence boundaries are
not sentence boundaries. The cause of this difference is most prominently the
fact that in the Brown corpus, a comma can sometimes end a sentence (such as in
poems). This also explains why the tokenizers score so high on the segmentation
of the Brown corpus.

\section{The Speed of the System}
\label{sec:eval-spd}

\chapter*{Conclusion}
\addcontentsline{toc}{chapter}{Conclusion}

We have presented a data-driven system for tokenizing and segmenting text. We
have demonstrated the system's versatility by combining methods based on
different techniques such as morphological dictionaries, regular expressions
and exception lists. The system proved its universal applicability in being
able to act both as a sentence boundary disambiguator for languages such as
English and Czech and as a word segmenter for languages which do not use
whitespace such as Chinese. We have also pointed to the fact that the program
relies only on multi-platform programs and libraries. While it has not been
tested on Windows or MacOS yet, care was taken at every step to ensure it would
be a smooth transition (ICU can be used instead of libiconv for character code
conversion, CMake is used for building, OS-specific matters are accessed via
Boost only\ldots).

We measured the accuracy, precision, recall and F-measure of the token and
sentence boundary disambiguation. The tests were executed with several very
different tokenization schemes and on several datasets in multiple languages.
We also measured and analyzed the tokenizer's speed and identified the
bottleneck which should serve as an avenue for further optimization.

The natural next step would be to invent and experiment with new ways and
features for tokenizing and segmenting text. The system offers fast feedback on
the accuracy of the user's tokenization schemes and is helpful in pointing out
positions in the text which are yet to be covered by rules for inserting
decision points. Another possible elaboration might be to change the maximum
entropy training back-end to the Toolkit for Advanced Discriminative Modelling
or some other alternative.


%\appendix
%\chapter{Automatic Detection of Irregular Annotation}
\label{chap:irreg}


%%% Seznam použité literatury
\nocite{*}
\bibliographystyle{iso690}
\cleardoublepage
\phantomsection
\addcontentsline{toc}{chapter}{Bibliography}
\bibliography{sbd,maxent,seg,web}

%%% Tabulky v bakalářské práci, existují-li.
\listoftables

%%% Použité zkratky v bakalářské práci, existují-li, včetně jejich vysvětlení.
%\chapwithtoc{List of Abbreviations}

%%% Přílohy k bakalářské práci, existují-li (různé dodatky jako výpisy
%%% programů, diagramy apod.). Každá příloha musí být alespoň jednou
%%% odkazována z vlastního textu práce. Přílohy se číslují.
\chapwithtoc{Attachments}

\input{tex/readme}

\openright
\end{document}
