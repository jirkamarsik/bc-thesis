%%% Hlavní soubor. Zde se definují základní parametry a odkazuje se na ostatní
%%% části. %%%

%% Verze pro jednostranný tisk:
% Okraje: levý 40mm, pravý 25mm, horní a dolní 25mm
% (ale pozor, LaTeX si sám přidává 1in)
\documentclass[12pt,a4paper]{report}
\setlength\textwidth{145mm}
\setlength\textheight{247mm}
\setlength\oddsidemargin{15mm}
\setlength\evensidemargin{15mm}
\setlength\topmargin{0mm}
\setlength\headsep{0mm}
\setlength\headheight{0mm}
% \openright zařídí, aby následující text začínal na pravé straně knihy
\let\openright=\clearpage

%% Pokud tiskneme oboustranně:
% \documentclass[12pt,a4paper,twoside,openright]{report}
% \setlength\textwidth{145mm}
% \setlength\textheight{247mm}
% \setlength\oddsidemargin{15mm}
% \setlength\evensidemargin{0mm}
% \setlength\topmargin{0mm}
% \setlength\headsep{0mm}
% \setlength\headheight{0mm}
% \let\openright=\cleardoublepage

%% Pokud používáte csLaTeX (doporučeno):
%\usepackage{czech}
%% Pokud nikoliv:
\usepackage[czech,english]{babel}
\selectlanguage{english}
\usepackage[T1]{fontenc}

%% Použité kódování znaků: obvykle latin2, cp1250 nebo utf8:
\usepackage[utf8]{inputenc}

%% Ostatní balíčky
\usepackage{graphicx}
\usepackage{amsthm}

%% Balíček hyperref, kterým jdou vyrábět klikací odkazy v PDF,
%% ale hlavně ho používáme k uložení metadat do PDF (včetně obsahu).
%% POZOR, nezapomeňte vyplnit jméno práce a autora.
\usepackage[ps2pdf,unicode]{hyperref}   % Musí být za všemi ostatními balíčky
\hypersetup{pdftitle=Fast and Trainable Tokenizer for Natural Languages}
\hypersetup{pdfauthor=Jiří Maršík}

%%% Drobné úpravy stylu

% Tato makra přesvědčují mírně ošklivým trikem LaTeX, aby hlavičky kapitol
% sázel příčetněji a nevynechával nad nimi spoustu místa. Směle ignorujte.
\makeatletter
\def\@makechapterhead#1{
  {\parindent \z@ \raggedright \normalfont
   \Huge\bfseries \thechapter. #1
   \par\nobreak
   \vskip 20\p@
}}
\def\@makeschapterhead#1{
  {\parindent \z@ \raggedright \normalfont
   \Huge\bfseries #1
   \par\nobreak
   \vskip 20\p@
}}
\makeatother

% Toto makro definuje kapitolu, která není očíslovaná, ale je uvedena v obsahu.
\def\chapwithtoc#1{
\chapter*{#1}
\addcontentsline{toc}{chapter}{#1}
}

\usepackage{mystyle}

\begin{document}

% Trochu volnější nastavení dělení slov, než je default.
\lefthyphenmin=2
\righthyphenmin=2


%%% Titulní strana práce

\pagestyle{empty}
\begin{center}

\large

Charles University in Prague

\medskip

Faculty of Mathematics and Physics

\vfill

{\bf\Large BACHELOR THESIS}

\vfill

\centerline{\mbox{\includegraphics[width=60mm]{img/logo.eps}}}

\vfill
\vspace{5mm}

{\LARGE Jiří Maršík}

\vspace{15mm}

% Název práce přesně podle zadání
{\LARGE\bfseries Fast and Trainable Tokenizer for Natural Languages}

\vfill

% Název katedry nebo ústavu, kde byla práce oficiálně zadána
% (dle Organizační struktury MFF UK)
Institute of Formal and Applied Linguistics

\vfill

\begin{tabular}{rl}

Supervisor of the bachelor thesis: & RNDr. Ondřej Bojar, Ph.D. \\
\noalign{\vspace{2mm}}
Study programme: & Computer Science \\
\noalign{\vspace{2mm}}
Specialization: & General Computer Science \\
\end{tabular}

\vfill

% Zde doplňte rok
Prague 2011

\end{center}

\newpage

%%% Následuje vevázaný list -- kopie podepsaného "Zadání bakalářské práce".
%%% Toto zadání NENÍ součástí elektronické verze práce, nescanovat.

%%% Na tomto místě mohou být napsána případná poděkování (vedoucímu práce,
%%% konzultantovi, tomu, kdo zapůjčil software, literaturu apod.)

\openright

\noindent
Thanks for all the fish!

\newpage

%%% Strana s čestným prohlášením k bakalářské práci

\vglue 0pt plus 1fill

\noindent
I declare that I carried out this bachelor thesis independently, and only with
the cited sources, literature and other professional sources.

\medskip\noindent
I understand that my work relates to the rights and obligations under the Act
No. 121/2000 Coll., the Copyright Act, as amended, in particular the fact that
the Charles University in Prague has the right to conclude a license agreement
on the use of this work as a school work pursuant to Section 60 paragraph 1 of
the Copyright Act.

\vspace{10mm}

\hbox{\hbox to 0.5\hsize{%
In ........ date ............
\hss}\hbox to 0.5\hsize{%
signature
\hss}}

\vspace{20mm}
\newpage

%%% Povinná informační strana bakalářské práce

\vbox to 0.5\vsize{
\setlength\parindent{0mm}
\setlength\parskip{5mm}

\begin{otherlanguage}{czech}

Název práce:
Rychlý a trénovatelný tokenizér pro přirozené jazyky
% přesně dle zadání

Autor:
Jiří Maršík

Katedra:  % Případně Ústav:
Ústav formální a aplikované lingvistiky
% dle Organizační struktury MFF UK

Vedoucí bakalářské práce:
RNDr. Ondřej Bojar Ph.D., Ústav formální a aplikované lingvistiky
% dle Organizační struktury MFF UK, případně plný název pracoviště mimo MFF UK

Abstrakt:
% abstrakt v rozsahu 80-200 slov; nejedná se však o opis zadání bakalářské
% práce

Klíčová slova:
% 3 až 5 klíčových slov

\end{otherlanguage}

\vss}\nobreak\vbox to 0.49\vsize{
\setlength\parindent{0mm}
\setlength\parskip{5mm}

Title:
Fast and Trainable Tokenizer for Natural Languages
% přesný překlad názvu práce v angličtině

Author:
Jiří Maršík

Department:
Institute of Formal and Applied Linguistics
% dle Organizační struktury MFF UK v angličtině

Supervisor:
RNDr. Ondřej Bojar Ph.D., Institute of Formal and Applied Linguistics
% dle Organizační struktury MFF UK, případně plný název pracoviště
% mimo MFF UK v angličtině

Abstract:
% abstrakt v rozsahu 80-200 slov v angličtině; nejedná se však o překlad
% zadání bakalářské práce

Keywords:
% 3 až 5 klíčových slov v angličtině

\vss}

\newpage


%%% Strana s automaticky generovaným obsahem bakalářské práce. U matematických
%%% prací je přípustné, aby seznam tabulek a zkratek, existují-li, byl umístěn
%%% na začátku práce, místo na jejím konci.

\openright
\pagestyle{plain}
\setcounter{page}{1}
\tableofcontents

%%% Jednotlivé kapitoly práce jsou pro přehlednost uloženy v samostatných
%%% souborech
\include{tex/introduction}
\chapter{Title of the first chapter}

\section{Title of the first subchapter in the first chapter}

\section{Title of the second subchapter in the first chapter }


\chapter{Title of the second chapter}

\section{Title of the first subchapter in the second chapter}

\section{Title of the second subchapter in the second chapter}



% Ukázka použití některých konstrukcí LateXu (odkomentujte, chcete-li)
%%% Ukázka použití některých konstrukcí LaTeXu

\begin{otherlanguage}{czech}

\subsection{Ukázka \LaTeX{}u}
\label{ssec:ukazka}

V~této krátké části ukážeme použití několika základních konstrukcí \LaTeX{}u,
které by se vám mohly při psaní práce hodit.

Třeba odrážky:

\begin{itemize}
\item Logo Matfyzu vidíme na obrázku~\ref{fig:mff}.
\item Tato subsekce má číslo~\ref{ssec:ukazka}.
\item Odkaz na literaturu~\cite{lamport94}.
\end{itemize}

Druhy pomlček:
červeno-černý (krátká),
strana 16--22 (střední),
$45-44$ (minus),
a~toto je --- jak se asi dalo čekat --- vložená věta ohraničená dlouhými pomlčkami.
(Všimněte si, že jsme za \verb|a| napsali vlnovku místo mezery: to aby se
tam nemohl rozdělit řádek.)

% Makro na české uvozovky (novější verze LaTeXu ho už mají zabudované)
%\newcommand{\uv}[1]{\quotedblbase #1\textquotedblleft}
\uv{České uvozovky.}

\newtheorem{theorem}{Věta}
\newtheorem*{define}{Definice}	% Definice nečíslujeme, proto "*"

\begin{define}
{\sl Strom} je souvislý graf bez kružnic.
\end{define}

\begin{theorem}
Tato věta neplatí.
\end{theorem}

\begin{proof}
Neplatné věty nemají důkaz.
\end{proof}

\begin{figure}
	\centering
	\includegraphics[width=30mm]{img/logo.eps}
	\caption{Logo MFF UK}
	\label{fig:mff}
\end{figure}

\end{otherlanguage}


\include{tex/conclusion}

%%% Seznam použité literatury
\include{tex/bibliography}

%%% Tabulky v bakalářské práci, existují-li.
\chapwithtoc{List of Tables}

%%% Použité zkratky v bakalářské práci, existují-li, včetně jejich vysvětlení.
\chapwithtoc{List of Abbreviations}

%%% Přílohy k bakalářské práci, existují-li (různé dodatky jako výpisy
%%% programů, diagramy apod.). Každá příloha musí být alespoň jednou
%%% odkazována z vlastního textu práce. Přílohy se číslují.
\chapwithtoc{Attachments}

\openright
\end{document}
